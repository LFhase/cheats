\input cheatmac


\def\DSPACE{{\rm DSPACE}}
\def\NSPACE{{\rm NSPACE}}
\def\DTIME{{\rm DTIME}}
\def\NTIME{{\rm NTIME}}
\def\PSPACE{{\rm PSPACE}}
\def\pr#1{{\csc #1}}

% TODO: define \pr to be problem name

\title{Complexity}
{\bf Note}: Strange structure due to finals topics.
Will improve later.

\dfn{A \term{deterministic Turing machine with $k$ tapes} $M_k$ is a tuple $(Σ,Q,δ,q_0,F)$.
$δ$ is the usual transition function, $Q$ the set of all states, $Σ$
language, $F$ accepting states, $q_0$ starting state.}

\dfn{A \term{nondeterministic Turing machine} is a machine such that $δ$ produces a set
of future states instead of just one state.}

{\bf Notation:} In this text, $L$ will always be a language, and $Φ, Ψ$ functions
as well as $f,g$. $M,N$ are DTM unless stated otherwise, $M_k$ with $k$ tapes.

\TODO{Blum complexity measure.}

\TODO{Recursive, computable notions.}

\fsection{Speedups and compressions}

\thm[Alphabet compression.]{If $f$ is a time-constructible function $f: {\msbm N} \Natu → \Natu$ with a TM with alphabet $Γ$, then $f$ is computable
in time $4 \log |Γ| T(n)$ with a 4-character alphabet TM.}

\prf{Use alphabet $0,1,→,|$. Encode ternary and higher objects into binary.}

\thm[Linear space compression]{ $L$ is accepted by $M_k$ with $S(n)$, then $∀r∈ℕ∃ N_k$ such that
$N_k$ accepts $L$ with $S_N(n) = ⌈S(n)/r⌉$.}

\prf{Since we do not care about time, we can encode a $r$-tuple of characters
into one character. We also need special characters so that we know on which character is
our tape. Moving may then be replaced with changing a a\c{b}c to
ab\c{c}.}

\res{$∀ r ∈ ℕ: \DSPACE(S(n)) = DSPACE(S(n)/r)$.}

\res{$∀ r ∈ ℕ: \NSPACE(S(n)) = NSPACE(S(n)/r)$.}

\thm[Tape reduction]{If we have a $M_k$ accepting $L$, then there is $M_1$ accepting $L$ with equal
$S(n)$.}

\prf{Again, we do not care about time, so we simply store all $k$ characters on one tape and and have special
characters for storing state of the emulated heads.}

\thm[Tape time reduction]{If we have a $M_k$ accepting $L$ in time $T(n)$, there is a $M_1$ accepting
it in time $5kT(n)^2$.}

\prf{Input is at the beginning of the 1 tape. Encode the other tapes after it. Create a larger alphabet with all characters having an
extra flag whether the head is on them or not. Sweep the tape twice,
first time load the marked characters to your ``registers'', the
second time do the transition.  }

\lem{Let $k≥2, r>0$. If we have a $M_k$ accepting $L$ with time complexity $T(n)$,
there is a $N$ that accepts $L$ in time $n + ⌈n/r⌉ + 6⌈t/r⌉]$.}

\prf{Assume we can write on the input tape. First, encode the input on the
first work tape and make it $r$ times denser. Then use input tape as
work tape. In every step, read one character to the left and one
character to the right of the current head position. This can be done
in 4 moves. In $r$ steps of the original $M_k$, only $r$ characters
can be changed from the current head position. The $M_k$ could only
modify two of the three positions, and we can edit them using $2$
steps. Of course, this increases the complexity of the transition
function.}


\thm[Linear Speedup]{Let $k≥2, ε>0$. Then if $M_k$ accepts $L$ in time $T(n) ≥ ω(n)$, there exists $N$ s.t.
it accepts $L$ in time $εT(n)$.}

\prf{Simply choose $r$ in the Lemma based on $ε$, usually things like $21/ε$.
}

% ---

\fsection{Constructible functions}

\dfn{A function $Φ: ℕ → ℕ$ is recursive if there is a TM that can compute $1^{f(n)}$ with input of size $1^n$.}

\dfn{A function $Φ$ is T/S-enumerable in T/S $O(f) ≡$ it is recursive and $ ∃$ TM such that the TM enumerates it
within $cf(n)$ steps for input of size $n$.}

\dfn{A function $Φ$ is T/S-constructible (time-constructible or space-constructible) $≡ ∃ $ a TM such that it halts
after exactly $Φ(n)$ steps for input of size $n$.}

\lem{Let $f_1 + f_2$ and $f_2$ be a T-constructible functions and $∃ε>0 ∃n_0 ∀ n≥ n_0 f_1(n) ≥ εf_2(n) + (1+ε)n$.
Then $f_1$ is time constructible also.}

\prf{Apply speedup to find a machine pacing exactly at $f_1$.}

\thm[Time Equivalence]{Let $f:ℕ→ℕ$ be a function s.t. $∃ε>0∃n_0∀n≥n_0: f(n) ≥ (1+ε)n$.
Then $f$ time-enumerable in time $O(f) ⇔ f$ time constructible.}

\prf{Backwards direction immediate, forwards direction by previous lemma.}

\thm[Space Equivalence]{Space-enumerability and constructibility coincide.}

\prf{We can use space compression to get it exactly, no lemmas required.}

\thm{Every time constructible $f$ is space constructible.}

% ---

\fsection{Complexity classes}

\TODO{DSPACE, DTIME}

\thm{$Φ$ recursive function $→ ∃$ recursive language $L ∉ DTIME/DSPACE(Φ(n))$.}
\prf{ Diagonal argument. Order machines, inputs $L ≡ \{x_i | M_i$ does not accept $x_i$ in time $Φ(|x_i|) \}$.}

\thm{
\itemize\ibull
\: $Φ, Ψ: ℕ → ℕ$ and $Ψ ∈ Ω(Φ)$ with $Ψ$ being space-constructible $→  ∃ L ∈ DSPACE(Ψ) ∖ DSPACE(Φ)$.
\: $Φ, Ψ: ℕ → ℕ$ and $Ψ ∈ Ω(Φ \log Φ)$ with $Ψ$ being time-constructible $→ ∃ L ∈ DTIME(Ψ) ∖ DTIME(Φ)$.
\endlist
}

\prf{ {\bf Part 1.} Create a machine $M$ which allocates $Ψ$ memory and accepts only if any single-tape machine
$M_w$ refuses $w$ in space $Ψ$. If this language was in $\DSPACE(Φ)$, there would be a machine $N$ recognizing this.
This machine $N$ can be assumed to be single-tape and always stops. We can emulate this machine, and so we have
a contradiction.

{\bf Part 2.} Similarly, only we use $Φ \log Φ$ for the tape reduction theorem. We set time to be so high that
the would-be machine can be emulated safely.
}

\thm[Relationship Theorem]{
\itemize\ibull
\: $\DTIME  ⊆ \NTIME$.
\: $\DSPACE ⊆ \NSPACE$.
\: $\DTIME  ⊆ \DSPACE$.
\: $\NTIME  ⊆ \DSPACE$ for s-c. functions.
\: $L ∈ \DSPACE(Φ(n)) ∧ Φ(n) ≥ \log n \then L ∈ \DTIME(c_L^{Φ(n)})$, with $c_L ≡ c(L)$. 
\: $L ∈ \NSPACE(Φ(n)) ∧ Φ(n) ≥ \log n \then L ∈ \DTIME(c_L^{Φ(n)})$, with $c_L ≡ c(L)$.
\: $L ∈ \NTIME(Φ(n)) \then L ∈ \DTIME(c_L^{Φ(n)})$ with $c_L ≡ c(L)$.
\endlist
}

\fsection{Savitch Theorem}

\thm[Savitch]{If $Φ(n) ≥ \log n$ and $Φ$ is space-constructible $\then NSPACE(Φ(n)) ⊆ DSPACE(Φ^2(n))$.}

\prf{USTCON is an NL-complete problem. (Basically, any problem in NL can be solved through graph search of
the state space.) We can solve USTCON by recursion -- basically, for every middle vertex $v$, we check whether
there is a $k/2$ $sv$-path and $k/2$ $vt$ path. Recursion takes $O(\log n)$ steps and we remember $O(\log n)$
at every step -- $O(\log^2 n)$ units of memory in total.}

\res{$PSPACE = NPSPACE$.}

\fsection{NP-complete problems}

\thm[Cook-Levin]{There exists a $NPc$ problem.}

\prf{Proved for $\pr{TILE}$. Simulate operations on the Turing machine tape by tiling.}

Standard transformations:

\pr{TILE} $∝$ \pr{SAT}: $x_{ijk} = 1 ≡$ position $(i,j)$ contains a tile of type $k$. Add clauses
that ensure validity of transformations of tiles.

\pr{SAT} $∝$ \pr{3-SAT}: Unwrap the long clauses into smaller ones using substitution.

\pr{3-SAT} $∝$ \pr{3-COLOR}: Set a triangle to define T/F colours. Connect all literal vertices to the
third colour. Use a gadget so that no three variables in the clause get F.

\pr{SAT} $∝$ \pr{CLIQUE}: Set vertices to literals so that every clique is an internally consistent
assignment.


\fsection{$\PSPACE$}
 
\dfn{$\PSPACE ≡$ a class of problems solvable with polynomial space.}

\dfn{$\pr{TQBF} ≡ $ language on a universe of fully quantified Boolean formulas, containing true ones.}

\thm{$\pr{TQBF}$ is a $\PSPACE$-complete problem.}

\prf{$\pr{TQBF} ∈ \PSPACE$: If it has $n$ variables and formula is of size $m$, evaluating it recursively uses $poly(n+m)$ memory. In fact, you can do it in $O(n+m)$ space since you restrict yourself on the same formula.

$\pr{TQBF}$ is complete: Transform every problem searching on a Turing machine to a $\pr{TQBF}$ in a similar
way like in the Savitch theorem. Specifically, there exists a SAT formula that can decide adjacency, use it
inductively to declare formulas that calculate $2^{i-1}$ adjacency. }

\fsection{Polynomial Hierarchy}

\dfn{$Σ^p_2$ is a class $x ∈ L ≡ ∃u ∈ {0,1}^{|q(x)|} ∀v ∈ {0,1}^{|q(x)|} M(x,u,v)=1$.}

Contains NP and coNP. EXACT INDSET example.

\dfn{$Π^p_2$ complement of $Σ_2^p$. Contains EXACT INDSET also.}

\dfn{$Σ^p_i$ and $Π^p_i$ defined analogously.}

\obs{If $P=NP$ then $PH=P$.}

\prf{Induction, simply use $P=NP$ to do away with the extra $∃$.}

\thm{If there is a PH-complete language, then PH collapses.}

\prf{(Sketch.) If there is such a language, it is a member of some class, and then the higher classes collapse.}

\thm{$Σ_i-SAT$ is a $Σ^p_i$-complete problem.}

\fsection{Pseudopolynomial alg., Strong NP-completeness}

\dfn{ An algorithm is {\it strongly NP-complete} $≡$ it is NP-complete even if all its input numerical variables
are bounded by the size of the input.}

\obs{Bin packing is strongly NP-complete.}

\obs{Subset sum, backpack problem and its related problems are not strongly NP complete, because they employ
pseudopolynomial algorithms.}

\fsection{\#P, \#P-completeness}

\dfn{$\#P ≡$ a class of enumerating problems (i.e. functions) where the associated decision problems are polynomial.}

\dfn{A function from $\{0,1\}*$ to $\{0,1\}*$  is in $FP ≡$ it can be computed by a TM in polynomial time.}

\opn{FP = \#P is an analog of P = NP.}

\dfn{A function $g$ is in $FP^f$ if it can be computed by a polynomial-time TM with oracle access to $f$.}

\dfn{A function $f$ is \#P-complete if it is in $\#P$ and every $g ∈ \#P$ is in $FP^f$. }

\thm{If \#CYCLE has a polynomial-time solution then $P=NP$. }

\thm{\#SAT is \#P-complete.}

Some \#P problems can be approximated easily, others not so much. Even easy problem like \#CYCLE or \#PM are hard.
\#PM is \#P-complete.

\bye
