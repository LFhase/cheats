% The UCW Macro Collection (a successor of mjmac.tex)
% Written by Martin Mares <mj@ucw.cz> in 2010 and placed into public domain
% -------------------------------------------------------------------------

%%% Prolog %%%

% We'll use internal macros of plain TeX
\catcode`@=11

%%% PDF output detection %%%

\newif\ifpdf
\pdffalse

\ifx\pdfoutput\undefined
\else\ifnum\pdfoutput>0
	\pdftrue
\fi\fi

%%% Page size and margins %%%

% If you modify these registers, call \setuppage afterwards
\newdimen\pagewidth
\newdimen\pageheight
\newdimen\leftmargin
\newdimen\rightmargin
\newdimen\topmargin
\newdimen\bottommargin
\newdimen\evenpageshift

\def\setuppage{%
	\hsize=\pagewidth
	\advance\hsize by -\leftmargin
	\advance\hsize by -\rightmargin
	\vsize=\pageheight
	\advance\vsize by -\topmargin
	\advance\vsize by -\bottommargin
	\hoffset=\leftmargin
	\advance\hoffset by -1truein
	\voffset=\topmargin
	\advance\voffset by -1truein
	\ifpdf
		\pdfpagewidth=\pagewidth
		\pdfpageheight=\pageheight
	\fi
}

% Set multiple margins to the same value
\def\sethmargins#1{\leftmargin=#1\relax\rightmargin=#1\relax\evenpageshift=0pt\relax}
\def\setvmargins#1{\topmargin=#1\relax\bottommargin=#1\relax}
\def\setmargins#1{\sethmargins{#1}\setvmargins{#1}}

% Define inner/outer margin instead of left/right
\def\setinneroutermargin#1#2{\leftmargin#1\relax\rightmargin#2\relax\evenpageshift=\rightmargin\advance\evenpageshift by -\leftmargin}

% Use a predefined paper format, calls \setuppage automagically
\def\setpaper#1{%
	\expandafter\let\expandafter\currentpaper\csname paper-#1\endcsname
	\ifx\currentpaper\relax
		\errmessage{Undefined paper format #1}
	\fi
	\currentpaper
}

% Switch to landscape orientation, calls \setuppage automagically
\def\landscape{%
	\dimen0=\pageheight
	\pageheight=\pagewidth
	\pagewidth=\dimen0
	\setuppage
}

% Common paper sizes
\def\defpaper#1#2#3{\expandafter\def\csname paper-#1\endcsname{\pagewidth=#2\pageheight=#3\setuppage}}
\defpaper{a3}{297truemm}{420truemm}
\defpaper{a4}{210truemm}{297truemm}
\defpaper{a5}{148truemm}{210truemm}
\defpaper{a6}{105truemm}{148truemm}
\defpaper{letter}{8.5truein}{11truein}
\defpaper{legal}{8.5truein}{14truein}

% Default page parameters
\setmargins{1truein}
\setpaper{a4}

%%% Placing material at specified coordinates %%%

% Set all dimensions of a given box register to zero
\def\smashbox#1{\ht#1=0pt \dp#1=0pt \wd#1=0pt}
\def\smashedhbox#1{{\setbox0=\hbox{#1}\smashbox0\box0}}
\def\smashedvbox#1{{\setbox0=\vbox{#1}\smashbox0\box0}}

% Variants of \llap and \rlap working equally on both sides and/or vertically
\def\hlap#1{\hbox to 0pt{\hss #1\hss}}
\def\vlap#1{\vbox to 0pt{\vss #1\vss}}
\def\clap#1{\vlap{\hlap{#1}}}

% \placeat{right}{down}{hmaterial} places <hmaterial>, so that its
% reference point lies at the given position wrt. the current ref point
\long\def\placeat#1#2#3{\smashedhbox{\hskip #1\lower #2\hbox{#3}}}

% Like \vbox, but with reference point in the upper left corner
\def\vhang#1{\vtop{\hrule height 0pt\relax #1}}

% Like \vhang, but respecting interline skips
\def\vhanglines#1{\vtop{\hbox to 0pt{}#1}}

% Crosshair with reference point in its center
\def\crosshair#1{\clap{\vrule height 0.2pt width #1}\clap{\vrule height #1 width 0.2pt}}

%%% Output routine %%%

\newbox\pageunderlays
\newbox\pageoverlays
\newbox\commonunderlays
\newbox\commonoverlays

% In addition to the normal page contents, you can define page overlays
% and underlays, which are zero-size vboxes positioned absolutely in the
% front / in the back of the normal material. Also, there are global
% versions of both which are not reset after every page.
\def\addlay#1#2{\setbox#1=\vbox{\ifvbox#1\box#1\fi\smashedvbox{#2}}}
\def\pageunderlay{\addlay\pageunderlays}
\def\pageoverlay{\addlay\pageoverlays}
\def\commonunderlay{\addlay\commonoverlays}
\def\commonoverlay{\addlay\commonoverlays}

% Our variation on \plainoutput, which manages inner/outer margins and overlays
\output{\ucwoutput}
\def\ucwoutput{\wigglepage\shipout\vbox{%
	\makeheadline
	\ifvbox\commonunderlays\copy\commonunderlays\nointerlineskip\fi
	\ifvbox\pageunderlays\box\pageunderlays\nointerlineskip\fi
	\pagebody
	\ifvbox\commonoverlays\vbox to 0pt{\vskip -\vsize\copy\commonoverlays}\fi
	\ifvbox\pageoverlays\vbox to 0pt{\vskip -\vsize\box\pageoverlays}\fi
	\makefootline
}\advancepageno
\ifnum\outputpenalty>-\@MM \else\dosupereject\fi}

\def\wigglepage{\ifodd\pageno\else\advance\hoffset by \evenpageshift\fi}

% Make it easier to redefine footline font (also, fix it so that OFS won't change it unless asked)
\let\footfont=\tenrm
\footline={\hss\footfont\folio\hss}

%%% Itemization %%%

% Default dimensions of itemized lists
\newdimen\itemindent		\itemindent=0.5in
\newdimen\itemnarrow		\itemnarrow=0.5in			% make lines narrower by this amount
\newskip\itemmarkerskip		\itemmarkerskip=0.4em			% between marker and the item
\newskip\preitemizeskip		\preitemizeskip=3pt plus 2pt minus 1pt	% before the list
\newskip\postitemizeskip	\postitemizeskip=3pt plus 2pt minus 1pt	% after the list
\newskip\interitemskip		\interitemskip=2pt plus 1pt minus 0.5pt	% between two items

% Analogues for nested lists
\newdimen\nesteditemindent	\nesteditemindent=0.25in
\newdimen\nesteditemnarrow	\nesteditemnarrow=0.25in
\newskip\prenesteditemizeskip	\prenesteditemizeskip=0pt
\newskip\postnesteditemizeskip	\postnesteditemizeskip=0pt

\newif\ifitems\itemsfalse
\newbox\itembox
\newcount\itemcount

\def\preitemize{
	\ifitems
		\vskip\prenesteditemizeskip
		\advance\leftskip by \nesteditemindent
		\advance\rightskip by \nesteditemnarrow
	\else
		\vskip\preitemizeskip
		\advance\leftskip by \itemindent
		\advance\rightskip by \itemnarrow
	\fi
	\parskip=\interitemskip
}

\def\postitemize{
	\ifitems
		\vskip\postnesteditemizeskip
	\else
		\vskip\postitemizeskip
	\fi
}

\def\inititemize{\begingroup\preitemize\itemstrue\parindent=0pt}

\def\itemize#1{\inititemize\setbox\itembox\llap{#1\hskip\itemmarkerskip}%
\let\:=\singleitem}

\def\singleitem{\par\leavevmode\copy\itembox\ignorespaces}

\def\endlist{\par\endgroup\postitemize}

\def\ibull{\raise0.2ex\hbox{$\bullet$}} % Signs frequently used for \itemize
\def\idot{\raise0.2ex\hbox{$\cdot$}}
\def\istar{\raise0.2ex\hbox{$\ast$}}

\def\numlist#1{\inititemize\itemcount=0\let\:=\numbereditem
\let\itemnumbering=#1}

\def\numbereditem{\par\leavevmode\advance\itemcount by 1
\llap{\itemnumbering\hskip\itemmarkerskip}\ignorespaces}

\def\nnorm{\the\itemcount}
\def\ndotted{\nnorm.}
\def\nparen{\nnorm)}
\def\nroman{\romannumeral\itemcount}
\def\nromanp{\nroman)}
\def\nalpha{\count@=96\advance\count@ by\itemcount\char\count@)}
\def\nAlpha{\count@=64\advance\count@ by\itemcount\char\count@)}
\def\ngreek{$\ifcase\itemcount\or\alpha\or\beta\or\gamma\or\delta\or\epsilon\or
\zeta\or\eta\or\theta\or\iota\or\kappa\or\lambda\or\mu\or\nu\or\xi\or\pi\or\rho
\or\sigma\or\tau\or\upsilon\or\phi\or\chi\or\psi\or\omega\fi$)}

%%% Miscellanea %%%

% {\I italic} with automatic italic correction
\def\I{\it\aftergroup\/}

% A breakable dash, to be repeated on the next line
\def\={\discretionary{-}{-}{-}}

% Non-breakable identifiers
\def\<#1>{\leavevmode\hbox{\I #1}}

% A handy shortcut
\let\>=\noindent

% Variants of \centerline, \leftline and \rightline, which are compatible with
% verbatim environments and other catcode hacks
\def\cline{\bgroup\def\linet@mp{\aftergroup\box\aftergroup0\aftergroup\egroup\hss\bgroup\aftergroup\hss\aftergroup\egroup}\afterassignment\linet@mp\setbox0\hbox to \hsize}
\def\lline{\bgroup\def\linet@mp{\aftergroup\box\aftergroup0\aftergroup\egroup\bgroup\aftergroup\hss\aftergroup\egroup}\afterassignment\linet@mp\setbox0\hbox to \hsize}
\def\rline{\bgroup\def\linet@mp{\aftergroup\box\aftergroup0\aftergroup\egroup\hss\bgroup\aftergroup\egroup}\afterassignment\linet@mp\setbox0\hbox to \hsize}

%%% Epilog %%%

% Let's hide all internal macros
\catcode`@=12
