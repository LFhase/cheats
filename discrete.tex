\input cheatmac
\def\vcorr{\vskip -2.5em}

\title{Diskrétní matematika}

\section{Pojmy}
\subsection{Množiny, relace}

\dfn[Potenční množina]{$2^X ≡$ systém všech podmnožin $X$.}

\dfn[Relace]{$R$ Relace na $X ≡$ podmnožina kartézského součinu $X \times X$.
Lidsky řečeno, nějaká množina uspořádaných dvojic z $X$.}

\dfn{Relace je \term{reflexivní}, pokud $∀x ∈ X: (x,x) ∈ R.$ (Zapisujeme $xRx$.)}

\dfn{Relace je \term{symetrická} $≡ ∀x,y ∈ X: (xRy) → (yRx)$.}

\dfn{Relace je \term{antisymetrická} $≡ ∀x,y ∈ X, x≠y: (xRy) → ¬(yRx)$ .}

\dfn{Relace je \term{tranzitivní} $≡ ∀x,y,z ∈ X: (xRy \andamp yRz) → (xRz)$.}

\dfn{Relace je \term{uspořádání} $≡$ je reflexivní, antisymetrická, tranzitivní.}

\dfn{Relace je \term{ekvivalence} $≡$ je reflexivní, symetrická, tranzitivní.}

\subsection{Uspořádání}

\nta{Většinou uspořádání značíme $(X,≤)$, čili píšeme $x≤y$ místo $xRy$.}

\dfn{\term{Řetězec} $C ⊆ X$ je množina prvků, ktere jsou všechny navzájem uspořádané,
čili $∀x,y ∈ C: x≤y ∨ y≤x$. Obvykle jsou uspořádané za sebou.}

\dfn{\term{Antiřetězec} $A ⊆ X$ je množina prvků, kde ani jeden není menší než jiný,
čili mezi sebou nemají žádné vztahy. $∀x,y ∈ A, x ≠ y: ¬(x≤y ∨ y ≤ x)$.}

Velikost největšího řetězce v uspořádání $S$ značíme $ω(S)$, velikost největšího
antiřetězce $α(S)$.

\thm[Dlouhý a široký]{Pro uspořádání $S=(X,≤)$ platí $α(S)ω(S) ≥ |X|$. 
}

\thm[Erdös-Szekerés]{Každá posloupnost $n^2+1$ reálných čísel obsahuje
monotónní posloupnost délky $n+1$.
}

\thm[Dilworth]{Každé uspořádání $S$ jde plně pokrýt pomocí $α(S)$
disjunktních řetězců.}

\subsection{Zobrazení}

\dfn{\term{Zobrazení} je jen jiný nazev pro funkci.}

V diskrétní matematice budou obvykle zobrazení z konečné množiny $M$
do konečné množiny $N$. Pozor, $N$ není obor hodnot, zobrazení $f(x) =
2$ může být třeba z $\{1,2,3\}$ do $\{1,2,3\}$.

\dfn{Zobrazení $f$ je \term{prosté} $≡ ∀x,y ∈ M, x≠y: f(x) ≠ f(y)$. }

\dfn{Zobrazení $f$ je \term{surjektivní} $≡ ∀y ∈ N ∃ x ∈ M: f(x) = y$.
Také se říká, že $f$ je \uv{na}.}

\dfn{Zobrazení $f$ je \term{bijekce} $≡$ $f$ je prosté a na.}
 
\subsection{Počítání}

\dfn[Faktoriál]{ $n! \equiv n \cdot (n-1) \cdots 1$.}

\dfn[Padající faktoriál]{ $n^{\underline{k}} \equiv n \cdot (n-1) \cdots (n-k+1)$.}

\dfn[Závorkové notace]{$[k]$ nejčastěji $\equiv \{1,2,\dots,k\}$, ale také $[n=1] \equiv 1$, pokud $n=1$, a $0$ jinak.}

\dfn[Kombinační čísla]{
$${n \choose k} \equiv {n! \over (n-k)! k!} = n^{\underline{k}} / k! = 
{n-1 \choose k} + {n-1 \choose k-1} $$}


\dfn[Stirlingova čísla 1. druhu]{
$${n \brack k} ≡ (n-1) {n-1 \brack k} + {n-1 \brack k-1}$$
}

\dfn[Stirlingova čísla 2. druhu]{
$${n \brace k} \equiv k { n-1 \brace k} + {n-1 \brace k-1}$$
\vcorr
$${n \brace k} = {1\over k!} \sum_{j=0}^k (-1)^j {k \choose j} (k-j)^n$$
}

\dfn[Bellova čísla (počet ekvivalencí)]{
$$B(n) \equiv \sum_{i=0}^n {n \brace i} $$
}

{\it Fibonacciho posloupnost:}
$$F_0 \equiv 0, F_1 \equiv 1, F_n \equiv F_{n-1} + F_{n-2}$$
\vcorr
$$F_n = {\varphi^n - (1-\varphi)^n \over \sqrt{5}}; \varphi = (1+\sqrt{5})/2.$$

{\it Harmonická posloupnost:}
$$H_0 \equiv 0, H_n \equiv \sum_{i=1}^n 1/i; H_n \approx \ln n$$

{\it Catalanova čísla/posloupnost:}
$$C_0 = 1 , C_1 = 1, C_n = ∑_{i=0}^n (C_i C_{n-i})$$
\vcorr
$$C_n = {1 \over n+1} {2n \choose n} ≈ {4^n \over n^{3/2} \sqrt{π}}.$$

{\it Binomická věta:}
$$(x+y)^n = {n \choose 0}x^n y^0 + {n \choose 1}x^{n-1} y^1 + \dots + {n \choose n}x^0 y^n$$
{\it Zobecněná binomická věta ($r$ reálné):}
$$(x+y)^r = \sum_{i=0}^{\infty} {r \choose i} x^{i}y^{r-i} $$

\subsection{Grafy}
$G = (V,E)$ graf. Hrany $(a,b) \rightarrow$ {\it orientovaný graf}.
{\it Multihrany} $\{a,b\}$, $\{a,b\}$ a {\it smyčky} $\{a,a\}$ se obvykle neuvažují.
Značení: $n = |G| = |V|$ počet vrcholů, $m = ||G|| = |E|$ počet hran.

{\it Stupeň} vrcholu je počet hran, které z~něj vedou -- list má například stupeň $1$.
Pro graf definujeme minimální stupeň $\delta$, průměrný stupeň $d$ (jako $2|E|/|V|$) a
maximální stupeň $\Delta$.

{\it Úplný graf} $K_n = ([n],{[n] \choose 2})$. Je-li orientovaný, říká
se mu {\it turnaj} (mezi 2 vrcholy turnaje je jen jedna or. hrana). Graf $H$ nazveme
{\it doplňkem} jiného grafu $G$ na stejné množině vrcholů,
má-li $H$ hrany mezi těmi vrcholy, mezi kterými je $G$ nemá, a naopak.

{\it Bipartitní graf} má dvě skupiny vrcholů, hrany jdou jen mezi skupinami (partitami).
{\it Úplný bipartitní} $K_{m,n}$ má partitu velikosti $m$, partitu velikosti $n$, a hrany
jsou všechny dvojice mezi nimi.

{\it Kružnice} $C_n$ má vrcholy zapojene do řetězu. {\it Cesta} $P_n$
je kružnice bez hrany. {\it Hamiltonovská cesta} v grafu $G$ je podgraf takový,
že je cestou na všech jeho vrcholech, čili podgraf $P_{|V(G)|}$. Ne všechny grafy
Hamiltonovskou cestu obsahují.

$G$ je {\it souvislý}, pokud se z~každého vrcholu dostanu po hranách do každého
jiného. Zobecněně: Graf je (hranově n. vrcholově) $(k+1)$-souvislý, pokud po odebrání
$k$ (hran n. vrcholů) je graf stále souvislý. Hrana n. vrchol je {\it kritická}, pokud jejich
odebrání sníží $k$-souvislost. $1$-kritická hrana je {\it most}, $1$-kritický vrchol
{\it artikulace}.

Graf $T$ je {\it strom}, pokud je souvislý a bez kružnice. Graf je {\it les}, pokud je
disj. sjednocení stromů. Každý strom je les, pro les platí $||T|| = |T| -c$, kde $c$
je počet komponent.

{\it Podgraf} $G'$ je nějaká podmnožina vrcholů grafu $G$ a nějaká
podmnožina těchto vybraných vrcholů. Podgraf je {\it indukovaný},
pokud vybereme všechny hrany, které mají oba konce v $G'$ a jsou
hranami $G$. Můžeme také říci, že graf $G$ obsahuje $I$
jako{indukovaný podgraf, pokud $I$ umím vytvořit z $G$ jen pomocí
odebírání vrcholů.

{\it Klika} je podgraf, který je také $K_i$ pro nějaké $i$. {\it Nezávislá množina}
je množina vrcholů, kde žádné dva spolu nesousedí. {\it Kostra} je největší podgraf co
do počtu vrcholů, který je zároveň stromem. Velikost největší kliky se značí
$\kappa(G)$, velikost největší nez. množiny $\alpha(G)$.

{\I Matice sousednosti} $A$ pro graf $G$ je taková matice,
kde počet sloupců i řádků je $|V|$ a dva vrcholy $i$ a $j$ spolu sousedí
hranou, právě když v~matici je v místě $A_{ij}$ jednička, na ostatních místech
jsou nuly.

{\I Matice incidence} $B$ pro graf $G$ je taková matice, kde počet sloupců je
$|E|$, počet řádků je $|V|$ a na pozici $A_ie$ je jednička, pokud vrchol $i$ je
jedním ze dvou konců hrany $e$, nula jinak.
Graf nazýváme \term{$k$-regulární}, pokud má všechny stupně stejné a rovnají se $k$.
Každá kružnice je $2$-regulární graf.

Řekneme o grafu, že je {\it $k$-obarvitelný}, pokud $∃$ funkce $c: V(G) →$ $\{1,2,… ,k\}$ taková, že každá hrana je dvoubarevná,
čili $∀uv ∈ E(G): c(u) ≠ c(v)$. {\it Barevnost grafu} $G$ je minimální $k$
takové, že graf $G$ je $k$-obarvitelný. Všimněte si, že $2$-obarvitelný
graf je bipartitní a naopak.

O grafu $G$, $|G|=n$ řekněme, že je {\it $d$-degenerovaný}, pokud pro něj
existuje (lineární) uspořádání všech vrcholů ($v_1,v_2, …, v_n)$ s
následující vlastností:

První vrchol $v_1$ v tomto uspořádání má stupeň nejvýše $d$. Vrchol
$v_2$ může mít v $G$ stupeň $d+1$, ale musí platit, že v grafu
$G - v_1$ (po odebrání vrcholu $v_1$) už má stupeň nejvýše $d$. Obecně
vrchol $v_i$ musí mít nejvýše stupeň $d$ v grafu $G-v_1 - v_2 - v_3 …
-v_{i-1}$.

\thm[Princip sudosti]{$∑_{v ∈ V} deg(v) = 2|E|$, čili specielně je součet
stup\-ňů sudé číslo.}

\thm[O bipartitních grafech]{Graf $G$ je bipartitní, právě když neobsahuje
lichou kružnici jako podgraf.}


\vfill\eject
% preferably on the second page
\section{Techniky}
\vskip -1.3em
\subsection{Matematická indukce}

Náhled na přirozená čísla:
\itemize\ibull
\: $0$ je přirozené číslo. ($1$ též.)
\: Pokud $x$ je přirozené čislo, $x+1$ je také přirozené číslo.
\endlist
Druhé pravidlo je ve formě implikace -- to, že $x$ je přirozené číslo,
máme \uv{zadarmo} jako předpoklad. Stejným způsobem se dá nahlížet na
cokoli, co má \uv{uvnitř} strukturu přirozených čísel.

{\it Příklad 1:} $\sum_{i=0}^n (2i+1) = (n+1)^2$.
\itemize\ibull
\: Základ indukce: $\sum_{i=0}^0 (2i+1) = 1 = (0+1)^2$.
\: Indukční krok: Dokazujeme implikaci. Předpokládem je, že vzorec $\sum_{i=0}^n (2i+1)$ platí pro všechna čísla od $0$ do $n$. Našim cílem je ukázat, že vzorec platí i pro hodnotu $n+1$.

Standardní postup: Situaci (vzorec) pro $n+1$ upravíme na vzorec pro $n$ a nějaký zbytek,
vzorec pro $n$ nahradíme pravou stranou pro $n$ (indukční předpoklad) a pravou stranu pro $n$
a zbytek upravíme na pravou stranu pro $n+1$. 
\endlist
\vskip -1.8em
$$ \sum_{i=0}^{n+1} (2i+1) = \left( \sum_{i=0}^n (2i+1) \right) + \left(2(n+1)+1\right) = $$
\vcorr
$$ = (n+1)^2 + (2n+3) = n^2 + 2n +1 + 2n + 3 = n^2 + 4n + 4 = (n+2)^2. $$
\vskip -1em
{\it Příklad 2:} Mějme rovinu a na ni $n$ přímek v obecné poloze
(žádné dvě nejsou rovnoběžné, žádné tři se neprotínají ve stejném
bodě). Pak tyto přímky rovinu sekají na ${n+1 \choose 2} + 1$ oblastí.

\itemize\ibull
\: Základ indukce: Jedna přímka seká rovinu na $2$ oblasti.
\: Indukční krok: Dokazujeme implikaci. Máme v rovině $n+1$ přímek v
obecné poloze. Víme, že když jednu odebereme, v rovině zbude $n$
přímek a dostaneme ${n+1 \choose 2}$ oblastí (indukční
předpoklad). Vrátíme tedy přímku do roviny, tato přímka má $n$
průsečíků z ostatními přímkami a tedy protíná $n+1$ oblastí napůl,
čili vznikne $n+1$ nových oblastí (zbytek). Sečteme:
\endlist
\vcorr
$${n+1 \choose 2} + 1 + (n+1) = {n^2 + n + 2n + 2 \over 2} + 1 = {n+2 \choose 2} + 1.$$ 
\vcorr
\subsection{Postup řešení sumačního příkladu}
\itemize\ibull
\: Zkusit malé případy, uhodnout výsledek, dokázat indukcí.
\: Rozložit sumu různými způsoby (a la $\sum k/2^k$).
\: Využít aritmetiku sum, vyměnit sumace.
\: Použít binomickou větu.
\: Když nic, tak aspoň vymyslet nějaké odhady.
\: Najít lineární rekurenci.
\: Použít vytvořující funkce.
\: Použít diskrétní integraci.
\endlist

\goodbreak
\subsection{Známé sumy}
\medskip
\halign{# \hfil & # \hfil\cr
$\sum_{i=0}^n 2^i = 2^{i+1}-1. $ & $\sum_{i=1}^n (2i-1) = n^2$ \cr
$\sum_{i=0}^n i2^i = (n-1)2^{n+1} + 2.$ & $\sum_{i=0}^n {n \choose i} = 2^n.$ \cr
$\sum_{i=r}^n {i \choose r} = {n+1 \choose r+1}.$ & $\sum_{i=0}^{k-1} {k+i \choose k-i-1} = F_{2k}. $ \cr
$\sum_{i=0}^{k} {k+i \choose k-i} = F_{2k+1}. $ & $\sum_{i=0}^n i^2 = n(n+1/2)(n+1)/3. $ \cr
$\sum_{i=0}^n i^3 = (\sum_{i=0}^n i)^2.$ & $\sum_{i=0}^\infty 1/2^i = 2. $ \cr
$\sum_{i=0}^\infty i/2^i = 2. $ \cr}

\subsection{Princip inkluze a exkluze}
Velikost sjednocení mů\-že\-me po\-čí\-tat pomocí postupného při\-čí\-tá\-ní a odčítání větších
a větších průniků (jejichž velikosti budou menší). Zapsáno matematicky:

$$|\bigcup_{1\dots n} A_i| = \sum_{j\ge 0} (-1)^j \sum_{I \in {1\dots n \choose j}} |\bigcap_I A_i|$$


{\it Přiklad: Spočítejte počet rozmístění 8 kamenů na šachovnici $4 × 4$ tak, aby žádné 4 neležely na stejném řádku nebo sloupci.}

Myšlenka inkluze a exkluze spočívá v tom, že umíme náš problém
rozdělit na nějaké části pomocí pravidelnosti. Obecně \term{jev}
(pojem z pravděpodobnosti) bude nějaká událost, kterou budeme chtít
spočítat nebo vyjádřit. Úkol se nás tedy ptá na:

$|A|,$ kde $A = \{x ; x$ je rozmístění 8 kamenů na šachovnici $4 × 4$
tak, že žádné 4 neleži na stejném řádku nebo sloupci$\}$.

$x$ je tedy onen jev, který chceme spočítat. Všimněme si, že ho můžeme
rozdělit na menší kusy pomocí logických spojek $∧$, $¬$ a $∨$:

$A = \{x ; x$ je rozmístění 8 kamenů tak že 4 nejsou na prvním řádku
$∧$ 4 nejsou na druhém řádku $∧$ 4 nejsou na třetím řádku \dots $\}$.

Vidíme, že naše jevy umíme rozložit na menší části pomocí operace
$∧$. Úplně stejně umíme rozložit i množinu $A$, akoráte pomocí operace
$∩$ (není náhoda, že vypadá jako $∧$:

Pokud $A_i ≡ \{x ; x $ je množina rozložení 8 kamenů tak, že 4 nejsou
na $i$-tém řádku $\}$ a $B_i ≡ \{x; x $ je množina rozložení 8 kamenů
tak, že 4 nejsou na $i$-tém sloupci $\}$, tak
\vskip -1em
$$A = A_1 ∩ A_2 ∩ A_3 ∩ A_4 ∩ B_1 ∩ B_2 ∩ B_3 ∩ B_4.$$
\vskip -1em
Poslední potíž máme s tím, že princip inkluze a exkluze počítá průniky
a ne sjednocení. Na to nam ale pomohou De Morganova pravidla, platí totiž:
\vskip -1em
$$\overline{A} = \overline{A_1} ∪ \overline{A_2} ∪ \overline{A_3} \dots $$
\vskip -1em
Pokud přecházíme z původního problému na problém opačný, často se
říká, že počítáme \term{špatné jevy} -- zde je vidět proč,
$\overline{A_1} = \{x; x$ je rozmístění 8 kamenů na šachovnici tak, že
4 kameny {\bf jsou} na prvním řádku.$\}$.

Velikosti jednotlivých špatných jevů spočítáme snadno úvahou:
$|\overline{A_i}| = |\overline{B_i}| = {12 \choose 4}$.

Protože kamenů je jen $8$, tak mohou být max. dva řádky nebo sloupce
najednou zaplněny. Z toho máme $|\overline{A_i} ∩ \overline{A_j}| =
|\overline{B_i} ∩ \overline{B_j}| = 1$ a velikosti průniků trojic a
výše už jsou nulové.  Ještě si rozmyslíme, že průnik řadku a sloupce
není nulový: $|\overline{A_i} ∩ \overline{B_j}| = {9 \choose 1}$.

Nyní aplikujeme princip inkluze a exkluze, jak ho známe:

$$|\overline{A}| = |⋃\overline{A_i}| = ∑_i |\overline{A_i}| + ∑_i |\overline{B_i}| - ∑_{i,j}|\overline{A_i} ∩ \overline{A_j}| -$$
\vskip -1.9em
$$ - ∑_{i,j} |\overline{B_i} ∩ \overline{B_j}| - ∑_{i,j}|\overline{A_i} ∩ \overline{B_j}| + 0 - 0 …$$

% \subsection{Diskrétní integrace}
% Místo derivace bereme diferenci, její ekvivalent v~konečném světě:
% $$ \Delta(f(x)) = f(x+1) - f(x).$$
% Potom platí, že pokud chceme vyřešit sumu 
% $ \sum_{i=a}^{b} g(i) $, pak musíme \uv{diskrétně integrovat} -- najít funkci, jejíž
% diference splňuje $\Delta(f(x)) = g(x)$. Pak vzorec pro sumu je $f(b+1) - f(a)$, podobně,
% jako je to v reálném integrálním počtu.

% Také víme, že diference $x^n$ se chová ošklivě, ale funkce $x^{\underline{n}}$ má diferenci
% podobnou, jako derivace $x^n$:
% $$\Delta(x^{\underline{n}}) = n x^{\underline{n-1}}.$$
% Z~tohoto faktu pak
% $$\sum_{i=0}^n i^{\underline{r}} = {(n+1)^{\underline{r+1}} \over r+1}.$$

\subsection{Grafové důkazy}

\subsubsection{Indukce}
Standardně se dělá podle počtu vrcholů nebo podle počtu hran.
Klasický průběh: odeberu vrchol nebo hranu, aplikuji na zbytek grafu indukci, vrchol nebo
hranu vrátím a zdůvodním, že dokazovaná věta platí i nadále.

Přiklad: Každý souvislý graf obsahuje alespoň $|V| -1$ hran.
Důkaz: Aplikujeme indukci podle poču hran.

Začátek indukce: Pokud souvislý graf neobsahuje kružnici, tak je stromem,
a tedy má přesně $|E| = |V| -1$ hran a tvrzení platí.

Indukční krok. Nechť graf kružnici obsahuje. Vezměme jednu hranu na
kružnici.  Hranu $e$ odebereme. Zbylý graf je souvislý, a tedy má
alespoň $|V|-1$ hran.  Přidáním hrany se počet hran jen zvýšil, a tedy
nerovnost $|E| ≥ |V| -1$ stále platí.

Kdy indukce nefunguje: pokud chceme něco dokazovat pro $k$-regulární
grafy, je těžké indukci použít. Nejmenší $k$-regulární graf je sice
$K_{k+1}$, ale to neznamená, že každý regulární graf jde vytvořit za
pomoci grafu $K_{k+1}$.

\subsubsection{Důkazy algoritmické}
Často se používá prohledávání do šířky nebo hloubky, případně hladový
algoritmus.

Příklad: Každý strom na alespoň 2 vrcholech obsahuje alespoň dva listy.

Důkaz: Začněme v libovolném vrcholu $v$. Pokud už je listem, tak si označíme
už jeden list za nalezený. Když ne, tak má stupeň alespoň $2$. Vydejme se prohledáváním
do hloubky, jednou po jedné jeho hraně a podruhé po druhé jeho hraně. Procházejme stromem,
dokud to jde.

Jakmile prohledávání už nemůže zpět, tak to nemůže být proto, protože z vrcholu
vedou hrany zpátky -- kdyby vedly, byla v grafu kružnice. Musí to být tedy proto, protože
z daného konečného vrcholu už není cesta zpět. Jeho stupeň tedy musí být jedna.

Protože jsme poslali prohledávání na dvě strany, dostaneme tak dva vrcholy se stupněm jedna
a tedy dva listy.

\subsubsection{Důkazy přímé a sporem}
Občas je dobré použít \uv{pseudoalgoritmický důkaz}: předpokládejme,
že objekt (nejdelší cesta, kružnice nějaké délky) je už nalezen, co to
potom říká o vlastnostech celého grafu? Výhodou je, že danou
cestu/kružnici nemusíme nalézt, ale \uv{spadne z nebe}.

Dalším užitečným trikem je zkusit něco dokazovat sporem, zvláště s
rovinností grafu nebo se sudostí/lichostí stupňů.

\bye
