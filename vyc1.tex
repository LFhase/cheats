\input macros.tex

\title{Vyčíslitelnost 1}
\section{Základní funkce}
$$\eqalign{ O(x) &= 0\cr
S(x) &= x+1\cr
I^j_n(x_1 \dots x_n) &= x_j\cr
S^m_n(f, g_{[1,m]}) &= h \st h(x_{[1,n]}) \simeq f(g_1(x_{[1,n]}) \dots g_m(x_{[1,n]})) \cr
R_n(f,g) &= h \st \cases{h(0,x_{[2,n]}) = f(x_{[2,n]}) \cr h(i, x_{[2,n]}) = g(i, h(i, x_{[2,n]}), x_{[2,n]})\cr} \cr
M_n(f) &= h \st \cr
h(x_{[1,n]}) &\daeq z \iff \cases{f(x_{[1,n]}, z) \daeq 0 \cr \forall j < z: f(x_{[1,n]}) \daneq 0\cr} \cr}$$

\itemize{\dfn{}}
\:  $ČRF$ -- lze ji odvodit pomocí základních funkcí
\:  $ORF$ -- $ČRF$ a zároveň totální (všude definovaná)
\:  $PRF$ -- lze ji odvodit bez použití minimalizace
\endlist

\obs{} Zneužití značení v taháku: občas $ČRF$ třída (píši $f \in ČRF$), občas zastupuje slova
{\I částečně rekurzivní funkce} (píši) $f\ ČRF$ n. $f {\rm\ je\ } ČRF$. Mělo by být jasné
z kontextu.

\thm{} $ČRF \subset ORF \subset PRF$

\prf{}

\itemize\ibull
\:  $ČRF \&\neg ORF$ je prázdná funkce $f$: $g(x,y) = y+1, f = M_1(g)$.
\:  $ORF \&\neg PRF$ je Ackermannova funkce. Dá se naprogramovat, totální též, tedy je $ORF$. Pokud však definujeme
úrovně výpočtu pro každé $k$, tak se dokážeme pomocí $k$ for cyklů ($PRF$) dostat jen do hloubky $k$ a nikdy to
tedy nepokryjeme celé.
\endlist

\thm{} $TM \simeq ČRF$, co se výpočetní síly týče.

\prf{} 

\thm{Kleene} $\forall k \ge 1:$

\numlist\nalpha
\: $\exists ČRF \Psi_k(e, \mls{x}{k})$ univerzální pro $k$ proměnných.
\: Lze efektivně získat z funkce její $e$ a z $e$ funkci.
\: $\exists PRF_1 U \& \exists PRF_{k+2} T_k \st$ \hfil\break
$\Psi_k(e, x_{[1,k]}) \simeq U(\mu_y(T_k(e, x_{[1,k]},y)))$
\: $\exists PRF_{m+1} s_m$ prostá a rostoucí $\st$ \hfil\break
$\Psi_{m+n}(e,y_{[1,m]},x{[1,n]}) \simeq \Psi_m\left(s_m(e,y_{[1,m]}), x_{[1,n]}\right)$
\: Univerzální funkce je standardní.
\endlist

\thm{} Univerzální $PRF$ není sama $PRF$, univerzální $ORF$ není sama $ORF$. Univerzální $PRF$ je v $ORF$.

\prf{} První dvě části diagonalizací.

\section{Rekurzivní spočetnost}
\itemize{\dfn{}}
\: $\chi \equiv$ pravdivostní funkce výroku nebo charakteristická funkce mny.
\: $PRP/ORP$ (predikáty) $\equiv$ jeho $\chi$ je $PRF/ORF$.
\: $RSP$ predikát $\equiv$ obor konvergence $ČRF$.
\: $M$ mna je rekurzivní $\equiv \chi \in ORF$.
\: $M$ mna je rekurzivně spočetná ($RSM$) $\equiv M = dom(f), f \in ČRF$.
\: $W_x = \left\{ m | \varphi_x(m) \da \right\}.$
\: $K = \left\{x | x \in W_x \right\}.$
\: $K_0 = \left\{ (y,x) | y \in W_x \right\}.$
\endlist

\thm{Post} $M$ rekurzivní $\iff$ $M$ i $\overline{M}$ jsou $RSM$. $P$ je $ORP$ $\iff$ $P$ a $\neg P$ jsou $RSP$.

\prf{} Tam: Když je $\chi \in ORF$, můžeme rozhodovat o přítomnosti i nepřítomnosti
efektivně. Zpátky: Prostě pustíme oba výpočty souběžně, jeden se zastaví, neboť jsou
obě $RSM$. Pro výroky se to získá z věty o selektoru (níže).

\thm{Vlastnosti $\Psi$}

\numlist\ndotted
\: $\Psi_k(e,x_{[1,k]})\da, \Psi_k(x_{[1,k]})\da$ jsou $RSP$, ale ne $ORP/PRP$.
\: $\neg\left(\Psi_k(e, x_{[1,k]})\da\right), \neg\left(\Psi_k(x_{[1,k]})\da\right)$ nejsou $RSP$.
\: $\Psi_k$ nelze rozšířit do $ORF$.
\endlist

\prf{}

\numlist\ndotted
\: Zřejmá.
\: $\Psi_1(x,x)\ua$ je $RSP \then \exists g \in ČRF$ co to počítá, tedy má kód $x_0$, diagonalizační spor.
\: Varianta důkazu, že neexistuje univerzální $PRF$.\
\endlist

\thm{O selektoru} $\forall RSP_{k+1} Q \exists ČRF_k \varphi \st$
$$\eqalign{
\Phi(x_{[1,n]})\da &\iff \exists y: Q\left(x_{[1,n]}, y\right)\cr
\Phi(x_{[1,n]})\da &\then Q\left(x_{[1,n]}, \varphi(x_{[1,n]})\right)\cr}$$

\prf{} Pro $Q \in RSP \exists T$ Turingův stroj takový, že $T$ se zastaví a přijme vstup $\iff$ $Q(x_{[1,n]},y)$.

\itemize{\res{}}
\: $\varphi \in ČRF \iff$ má rek. spočetný graf.
\: $\forall RSM$ je oborem hodnot nějaké $ČRF$.
\: $\forall f \in ČRF: \rng(f) \in RSM$.
\endlist

\fakesection{Převoditelnost}
\itemize{\dfn{}}
\: $A \le_m B \equiv \exists f, f \in ORF \st x \in A \iff f(x) \in B$.
\: $A \le_1 B \equiv f$ navíc prostá.
\: $M$ 1-úplná $\equiv$ $M \in RSM$ a $\forall B \in RSM: B \le_1 M$.
\: $M$ m-úplná $\equiv$ $M \in RSM$ a $\forall B \in RSM: B \le_m M$.
\endlist

\thm{Vlastnosti $K$} $K \in RSM$, $K$ není rekurzivní, $K$ je 1-úplná.

\prf{} První dvě vlastnosti zřejmé, pro třetí vezměme $W_x$ a $\alpha(y,x,w)$,
která ji popisuje ($w$ volná). Použijeme s-m-n větu, získejme $h(y,x) = s_2(e,y,x)$. Pak
$\forall y \in W_x:$

$$ \alpha(x,y,w)\da \iff \varphi_{h(y,x)}(w)\da \iff  \varphi_{h(y,x)}(h(y,x))\da \iff h(y,x) \in K.$$

\thm{Vlastnosti $K_0$} $K_0$ je 1-úplná.

\prf{} Převodem $K$ na $K_0$.

\section{Generování RSM}

\dfn{} $f$ úseková $\equiv \dom(f) = [0,n]$.

\obs{} $PRF, ORF$ jsou úsekové vždy.

\thm{Generátory $PRM$} Rekurzivní množiny jsou právě $\rng$ rostoucích úsekových $ČRF$. 

\prf{}

\itemize\ibull
\: \uv{$\then$}: očísluj prvky rek. množiny, $f(0) = \mu_x(x \in M)$, $f(k+1) = \mu_y(y > f(k), y \in M)$.
\: \uv{$\leftarrow$}: Pokud $f$ není totální, tak ji konečným cyklem umíme
vyčíslit, $\rng(f)$ je rekurzivní.  Pokud $f$ totální je, tak musíme pro $y$
najít $x \st f(x) = y$ bez minimalizace. Jenže víme, že díky úsekovosti a růstu
platí, že $\exists x: x \le f(x)$. Tak prostě najdeme horní mez pomocí $y$ a
pak minimalizujeme.
\endlist

\thm{} Rekurzivně spočetné množiny jsou právě obory hodnot prostých úsekových $ČRF$.

\prf{} Vyrob z $\varphi$ množinu $B = \{(x,s) | \varphi(x)$ vydá $x \in M$ přesně po $s$ krocích $\}$.
Ta je rekurzivní, takže má generátor rostoucí úsekovou $f$. Zadefinujeme $g(j) = (f(j))_{2,1}$
(první složka). Funkce $g$ je úseková a prostá, neboť ke každému $x$ existuje nejvýše jedno $s$.

\thm{} Každá nekonečná $RSM$ obsahuje nekonečnou rostoucí podmnožinu.
 
\fakesection{Imunní a simple množiny}

\dfn{} $M$ imunní $\equiv |M| \ge \omega \& W_x \subseteq M \then |W_x| < \omega$. 

\dfn{} $X$ simple $\equiv X \in RSM \& \overline{X}$ imunní.

\obs{Neefektivní konstrukce imunní množiny} Procházej prvky $W_x$ pro každé $x$ a přidávej elementy do $\overline{M}$.

\obs{Efektivní konstrukce simple množiny} Definujme predikát $Q$:
$$ Q(x,y) \iff y \in W_x \& y > 2x.$$
Tvrdíme, že $A = \rng(\varphi)$ je simple:

\itemize\ibull
\: Obor hodnot $ČRF$ je $RSM$.
\: $\varphi(x) > 2x$, pokud $\varphi(x) \da \then$ z množiny $[0,2x]$ jsme do $A$ dali nejvýš $x$ čísel $\then \overline{A}$
je nekonečná.
\: z konstrukce plyne: nepřidali jsme žádnou nekonečnou $RSM$ do $\overline{A}$.
\endlist

$\overline{A}$ je tedy imunní.

\fakesection{Věty o rekurzi}

\thm{První} $f \in ČRF_1 \then \exists a \st \forall x:$
$$\varphi_{f(a)}(x) \daeq \varphi_a(x).$$

\prf{} Hledané $a$ bude tvaru $a = s_1(z,z)$. Protože $s_1 \in PRF$, tak $\varphi_{f(s_1(z,x))}(x)$
je funkcí jen $z,x \then \exists e \st:$
$$\varphi_{f(s_1(z,z))}(x) \simeq \Psi_2(e,z,x) \simeq \Psi_1(s_1(e,z),x) \simeq \varphi_{s_1(e,z)}(x).$$
Dosaď za $z = e$, $a = s_1(e,e)$.
\goodbreak
\thm{Druhá} $f \in ČRF_1 \then \exists {\rm\ rostoucí\ } g \in PRF_1 \st \forall x:$
$$ \varphi_{f(g(j))}(x) \simeq \varphi_{g(j)}(x).$$

\prf{} $g(j) = s_2(z,z,j)$, vol $z = e$, podobně jako první věta.
\goodbreak
\thm{Třetí, pro více proměnných}
$f \in ČRF_{n+1} \then \exists a \forall x_{[1,n]}:$
$$ \varphi_a(x_{[1,n]}) \simeq f(a,x_{[1,n]}).$$

\prf{}
\itemize\ibull
\: Jedna proměnná: $$f(y,x) \simeq \Psi_2(e,y,x) \simeq \Psi_1(s_1(e,y),x) \simeq \varphi_{s_1(e,y)}(x).$$
Vol $g(y) = s_1(e,y)$, použij první větu.
\: Více proměnných:
$$\eqalign{
f(y,\mls x n ) &\simeq \Psi_{n+1}(e,y,\mls x n ) \simeq \Psi_n(s_1(e,y), \mls x n )\cr
\Psi_n(s_1(e,y), \mls x n ) &\simeq \varphi_{s_1(e,y)}(\mls x n ).\cr}$$
$s_1(e,y)$ má pevný bod $a$, vol $y = a$.
\endlist

\thm{Čtvrtá} $f \in ČRF_{n+1} \then \exists g \in PRF_n \st \forall x,\mls y n :$
$$ \varphi_{f(g(\mls y n ), \mls y n )}(x) \simeq \varphi_{g(\mls y n )}(x).$$

\prf{} Vol $g(\mls y n ) = s_{n+1}(z,z,\mls y n )$, zopakuj postup stejný, jako výše.

\thm{Rice} $A$ buď netriviální třída $ČRF_1 \then B = \{x | \varphi_x \in A\}$ není rekurzivní.

\prf{} Sporem, existuje nutně $b \in B$, $c \not \in B$, zadefinuj $f(x) = b$ pokud $x \not \in B$, $c$ jinak.
$f(x)$ je $ORF \then \exists e \st \varphi_{f(e)}(x) \simeq \varphi_e(x)$, a tedy $e \in B$ a $e \not \in B$.

\res{} Nelze efektivně rozhodnout ekvivalence programů.

\fakesection{Produktivní a kreativní množiny}

\dfn{} $B$ je produktivní $\equiv \exists \Phi \in ČRF \st \forall x: W_x \subseteq B \then \Phi(x) \da \& \Phi(x) \in B \ W_x$.

\dfn{} Funkci $\Phi$, která dokazuje produktivitu pro $B$, také říkáme produktivní.

\dfn{} $A$ kreativní $\equiv A$ $RSM \& \overline{A}$ je produktivní.

\obs{} $K$ je kreativní. $\Phi$ by byla identita.

\obs{} $f$ prostá a $ORF$, pak $\{x | f(x) \not \in W_x\}$ produktivní

\obs{} Každá produktivní množina obsahuje nekonečnou $RSM$. 

\thm{O produktivní $ORF$} $\forall B$ produktivní $\exists$ produktivní $f \in ORF$.

\prf{}
$B$ je produktivní, $\Phi$ přiřazená produktivní funkce. Výsledkem (jako obvykle) bude
nějaká funkce $f \equiv \varphi_{g(x)}$ pro vhodné $g$.

Nejprve pomocná funkce $\alpha(x,y,z)$: $\alpha(x,y,z)\da \iff \Phi(x)\da \& z \in W_y$.
Použijeme $s-m-n$ větu, dostaneme:
$$ \alpha(x,y,z) \simeq \Psi_3(e,x,y,z) \simeq \Psi_1(s_2(e,z,y),z).$$
Zadefinuj $PRF$ $\beta(x,y) \equiv s_2(e,x,y)$. Pak platí:
$$ \Psi_1(\beta(x,y),z) \da \iff \Phi(x)\da \& z \in W_y. $$
Co je $W_{\beta(x,y)}$? Pokud $\Phi(x) \ua$, tak $\emptyset$, jinak $W_y$.

Nyní aplikuj čtvrtou větu o rekurzi:
$$ \exists g \in PRF: \Psi_1(\Beta(g(y),y),z) \simeq \Psi_1(g(y),z). $$
$W_g(y)$ je tedy $W_{\beta(g(y),y)}$. Když $\Psi(g(y)) \da$, tak je to rovno $W_y$. 
Sporem dokáži, že $\Psi(g(y))$ nemůže nikdy divergovat.
 
\thm{O produktivní $ORF$ bijekci}
Každá produktivní množina $B$ má i příslušnou produktivní funkci, která je v $ORF$ a je bijekcí.

\prf{} Z minulé věty máme $ORF$ $f$ pro příslušnou $B$. Tu nejdřív přepočítáme na surjektivní $g$,
a nakonec i na injektivní $h$.

\itemize\ibull
\: V prvním kroku vytvoříme nekonečnou $M \st x \in M \then W_x = \omega$. Volíme programy, které
se zastaví na každý vstup.
\: V druhém kroku zadefinujeme $g(x)$ jako $g(x) = f(x) | x \not \in M$ a nebo $g(x) = j$, pokud $x$ je $j$-tý
prvek $M$. 
\: V třetím kroku definujeme $h(x)$ jako $g(x)$, pokud neporušíme prostotu. Když ji porušíme, najdeme index
$y_1$ množiny $W_{y_1} = W_x \cup \{g(x)\}$ a podíváme se na $g(y_1)$. 

\numlist\ndotted
\: Pokud $W_{y_1} \subseteq B$, tak $g(y_1) \in B \backslash W_{y_1}$. Kdyby i $g(y_1)$ se trefilo do předchozích $h(x)$, přidej předchozí $h(x)$ do $W_{y_1}$, vytvoř $y_2$ a opakuj, dokud se nedostaneš mimo předchozí hodnoty $h(x)$.
\: Pokud $W_{y_1} \not \subseteq B$, tak zvol $h(x)$ libovolně z nových hodnot.
\endlist
\endlist

\thm{O ekvivalenci pojmů} $\overline{K} \le_1 B \iff \overline{K} \le_m B \iff B$ je produktivní.

\prf{}
\itemize\ibull
\: $1 \then 2$ jasné.
\: $2 \then 3$: Ukážeme, že produktivita se zachovává vzhůru. Máme $C \le_m B$, čili
$x \in C \iff h(x) \in B$. Je-li $W_x \subseteq B$, tak $h^{-1}(W_x) \subseteq C$. Sestrojíme
tedy $\beta(x,y)$ tak, aby $\beta(x,y)\da \iff h(y) \in W_x$.
\: $3 \then 1$: Hledáme prostou $h \in ORF \st x \in \overline{K} \iff h(x) \in B$.
Zadefinuj $\beta(y,x,w) \da \iff w = f(y) \& x \in K$. Použij $s-m-n$ větu: $\alpha(y,x) = s_2(e,y,x) \then 
W_{\alpha(y,x)} = \{ w | f(w) = f(y) \& x \in K \}$. 
Nyní použij čtvrtou větu o rekurzi: $\exists g(x) \st \varphi_{g(x}(z) \simeq \varphi_{\alpha(g(x),x)}(z)$.
\endlist

\fakesection{Neoddělitelné množiny}

\dfn{} $A,B$ jsou rekurzivně neoddělitelné $\equiv$ $\neg\exists$ rekurzivní $M \st A \subseteq M \& B \subseteq \overline{M}$.

\dfn{} $A,B$ jsou efektivně neoddělitelné $\exists ČRF f \st:$
$$(A \subseteq W_x \& B \subseteq W_y \& W_x \cap W_y = \emptyset) \then f(x,y)\da \& f(x,y) \not \in W_x \cup W_y. $$

\obs{} Efektivně neoddělitelné množiny jsou rek. neoddělitelné. 

\prf{} Mám-li rekurzivní množinu $M$, pak snadno najdu funkce tak, aby $W_x = M$ a $W_y = \overline{M}$ a tedy $f$
nemůže existovat.

\thm{} Rek. neoddělitelné množiny nemusí být efektivně neoddělitelné.

\thm{} Existují dvě efektivně neoddělitelné množiny, a to:
$A = \{x | \varphi_x(x) \simeq 0 \}, B = \{ x | \varphi_x(x) \simeq 1 \}$.

\prf{} Mějme $A \subseteq W_x$, $B \subseteq W_y$. Vyrobíme $\varphi_{\alpha(x,y)}(w)$ divergující, když $w \not \in A \cup B$.
Úvahou plus použitím $s-m-n$ věty víme, že $\exists \varphi_{\alpha(x,y)}(w)$ taková, že vrátí $1$, pokud $w$ padne do $W_x$ dříve než do $W_y$, $0$ naopak a jinak diverguje. Diagonalizací vidíme, že $\varphi_{\alpha(x,y)}(\alpha(x,y))$ diverguje.

\obs{} Efektivně neoddělitelné $RSM$ $A$ a $B$ jsou kreativní.

\thm{Dvojná věta o rekurzi} Pro $ORF$ $f,g$ existují $m,n \st \varphi_m \simeq \varphi_{f(m,n)} \& \varphi_n \simeq \varphi_{g(m,n)}$.

\prf{} $\varphi_{f(x,y)} \then$ VR4 $\then \exists \alpha \st \varphi_{f(\alpha(y),y)} \simeq \varphi_{\alpha(y)}$.
$\varphi_{g(\alpha(y),y)} \then$ VR1 $\then \exists n \st \varphi_{g(\alpha(n), n)} \simeq \varphi_m$. Vol $m = \alpha(n)$.

\dfn{1-převoditelnost} $(A,B),(C,D)$ disj. dvojice, pak $(A,B) \le_1 (C,D) \equiv$ $A \le_1 C$ pomocí $h$ a $B \le_1 D$ pomocí té samé $h$.

\thm{Ekvivalence $1$-úplnosti a efektivní neoddělitelnosti} $(A,B)$ je $1$-úplná $\iff (A,B)$ efekt. neodd.

\prf{}

\itemize\ibull
\: \uv{$\then$}: $(A,B)$ 1-úplné, mohu zvolit $(C,D)$ efektivně neoddělitelné,
že $(C,D) \le_1 (A,B)$. Převodem přes $h$ tam a zpět získám funkci dokazující
neodd. $(A,B)$.

\: \uv{$\leftarrow$}: Máme $(C,D)$ efekt. neodd., což dokazuje $f$, a nějaké
$(A,B)$. Zadefinuj $W_{\gamma_1(x)}$ a $W_{\gamma_2(x)}$ takto:
$$\eqalign{
x \in D &\then W_{\gamma_1(x)} = A \cup \{f(\gamma_1(x), \gamma_2(x))\} \& W_{\gamma_2(x)} = B, \cr
x \in C &\then W_{\gamma_2(x)} = B \cup \{f(\gamma_1(x), \gamma_2(x))\} \& W_{\gamma_1(x)} = A. \cr}$$

Pokud $x \not \in C \cup D$ pak $f(\gamma_1(x), \gamma_2(x)) \not \in A \cup
B$.  Pokud $x \in C$, tak $f(\gamma_1(x), \gamma_2(x)) \in A$. Pro $x \in D$
podobně. Složení $f$ s $\gamma_{\{1,2\}}$ tedy dokazuje $1$-úplnost.
\endlist

\section{Gödelova věta}
\uv{V rozumných teoriích je množina dokazatelných a vyvratitelných formulí neoddělitelná.}

\dfn{} Teorie $T$ je axiomatizovatelná $\equiv$ množina dokazatelných formulí v $T$ je $RSM$. 

\thm{Gödel} Jestliže teorie $T$ 1. řádu je $ZAS$ a je bezesporná:

\itemize\ibull
\: Množina dokazatalných formulí není rekurzivní.
\: je-li $T$ axiomatizovatelná, tak $\exists$ formule $f$ nerozhodnutelná v $T$ a nelze v $T$ rozhodnout vlastní bezespornost.
\endlist

\section{Reprezentovatelnost}

\dfn{} $f \in ČRF$ je reprezentovatelná $\equiv$ $f(\mls x n ) = y \iff \vdash F( \mls x n , y)$ a také
$\vdash F(\mls x n , q) \& \vdash F(\mls x n , j) \then q = j$.

\dfn{} $P$ predikát je diofantický $\equiv \exists p_1, p_2$ polynomy $\st \forall \mls x n :$
$$ \exists \mls y k P(\mls x n ) \iff p_1(\mls x n , \mls y k ) = p_2(\mls x n , \mls y k ).$$

\thm{Matjasevič} $P$ diofantický $\iff$ $P$ rek. spočetný.

\prf{} Zpětná implikace je těžká a neříká se, dopředná naopak jednoduchá -- zkoušíme všechny možnosti.

\thm{} $\forall f \in ČRF, f$ je reprezentovatelná, dokonce $\exists$ formule, která ji reprezentuje ve všech teoriích.

\prf{} Podle Matjaseviče.

\res{}  Jsou-li $A$, $B$ disj. $RSM$, pak $\exists G$ formule v $\sigma_1 \st$ \dots

{\tt Verze \versionnumber, chyby zaručeny.} 
\bye
