\pdfoutput=1
\pdfpkresolution=600

\input ucwmac.tex
\input ucw-ofs.tex
\input color

\language\czech
%**start of header
\newcount\columnsperpage

\columnsperpage=3

\def\versionnumber{1.5}  % Version of this reference card
\def\year{2007}
\def\month{January}
\def\version{\month\ \year\ v\versionnumber}


\def\shortcopyrightnotice{
}

\def\copyrightnotice{
}

% make \bye not \outer so that the \def\bye in the \else clause below
% can be scanned without complaint.
\def\bye{\par\vfill\supereject\end}

\newdimen\intercolumnskip
\newbox\columna
\newbox\columnb

\def\ncolumns{\the\columnsperpage}

\message{[\ncolumns\space
   column\if 1\ncolumns\else s\fi\space per page]}

\def\scaledmag#1{ scaled \magstep #1}

% This multi-way format was designed by Stephen Gildea
% October 1986.

% Modified for Czech fonts.
\if 1\ncolumns
   \hsize 4in
   \vsize 10in
   \voffset -.7in
   \font\titlefont=\fontname\tenbf \scaledmag3
   \font\headingfont=\fontname\tenbf \scaledmag2
   \font\smallfont=\fontname\sevenrm
   \font\smallsy=\fontname\sevensy

   \footline{\hss\folio}
   \def\makefootline{\baselineskip10pt\hsize6.5in\line{\the\footline}}
\else
   \hsize 3.2in
   \vsize 7.95in
   \hoffset -.75in
   \voffset -.745in
   \font\titlefont=csbx10 \scaledmag2
   \font\headingfont=csbx10 \scaledmag1
   \font\smallfont=csr6
   \font\smallsy=cssy6
   \font\eightrm=csr8
   \font\eighti=csmi8
   \font\eightsy=cssy8
   \font\eightbf=csbx8
   \font\eighttt=cstt8
   \font\eightit=csti8
   \font\eightsl=cssl8
   \textfont0=\eightrm
   \textfont1=\eighti
   \textfont2=\eightsy
   \def\rm{\eightrm}
   \def\bf{\eightbf}
   \def\tt{\eighttt}
   \def\it{\eightit}
   \def\sl{\eightsl}
   \normalbaselineskip=.8\normalbaselineskip
   \normallineskip=.8\normallineskip
   \normallineskiplimit=.8\normallineskiplimit
   \normalbaselines\rm          %make definitions take effect

   \if 2\ncolumns
     \let\maxcolumn=b
     \footline{\hss\rm\folio\hss}
     \def\makefootline{\vskip 2in \hsize=6.86in\line{\the\footline}}
   \else \if 3\ncolumns
     \let\maxcolumn=c
     \nopagenumbers
   \else
     \errhelp{You must set \columnsperpage equal to 1, 2, or 3.}
     \errmessage{Illegal number of columns per page}
   \fi\fi

   \intercolumnskip=.46in
   \def\abc{a}
   \output={%
       % This next line is useful when designing the layout.
       %\immediate\write16{Column \folio\abc\space starts with \firstmark}
       \if \maxcolumn\abc \multicolumnformat \global\def\abc{a}
       \else\if a\abc
        \global\setbox\columna\columnbox \global\def\abc{b}
         %% in case we never use \columnb (two-column mode)
         \global\setbox\columnb\hbox to -\intercolumnskip{}
       \else
        \global\setbox\columnb\columnbox \global\def\abc{c}\fi\fi}
   \def\multicolumnformat{\shipout\vbox{\makeheadline
       \hbox{\box\columna\hskip\intercolumnskip
         \box\columnb\hskip\intercolumnskip\columnbox}
       \makefootline}\advancepageno}
   \def\columnbox{\leftline{\pagebody}}

   \def\bye{\par\vfill\supereject
     \if a\abc \else\null\vfill\eject\fi
     \if a\abc \else\null\vfill\eject\fi
     \end}
\fi

% ***** Verbatim typesetting *****

% Verbatim typesetting is done by
%    \verbatim"stuff to verbatim typeset"
% Any character can be used in place of ".
% E.g. \verbatim?stuff? or \verbatim|stuff|.  Cf. TeXbook pp.380-382

\def\uncatcodespecials{\def\do##1{\catcode`##1=12}\dospecials}
\def\setupverbatim{\tt%
\def\par{\leavevmode\endgraf}\catcode`\`=\active%
\obeylines\uncatcodespecials\obeyspaces}
\def\verbatim{\begingroup\setupverbatim\doverbatim}
\def\doverbatim#1{\def\next##1#1{##1\endgroup}\next}

\def\\{\verbatim}
\def\ds{\displaystyle}
\def\SPC{\quad} % space between symbol and command

\parindent 0pt
\parskip 1ex plus .5ex minus .5ex

\def\small{\smallfont\textfont2=\smallsy\baselineskip=.8\baselineskip}

\outer\def\newcolumn{\vfill\eject}

\outer\def\title#1{{\titlefont\centerline{#1}}\vskip 1ex plus .5ex minus.5ex}

%\outer\def\section#1{\par\filbreak
%  \vskip 1ex plus 2ex minus 2ex {\headingfont #1}\mark{#1}%
%  \vskip 1ex plus 1ex minus .5ex}
\outer\def\section#1{\par\filbreak
   \vskip .75ex plus 1ex minus 2ex {\headingfont #1}\mark{#1}%
   \vskip .5ex plus .5ex minus .5ex}
\outer\def\subsection#1{\par\filbreak
   \vskip .75ex plus 1ex minus 2ex {\bf #1}\mark{#1}%
   \vskip .5ex plus .5ex minus .5ex}

\def\paralign{\vskip\parskip\halign}

%\def\<#1>{$\langle${\rm #1}$\rangle$}

\def\begintext{\par\leavevmode\begingroup\parskip0pt\rm}
\def\endtext{\endgroup}

% smaller indenting of itemize lists for a compact format
\itemindent=0.15in
\itemnarrow=0in

% math-related definitions (compressed than usual):

\def\dfn{
{\color{red} \bf D:}
}


\def\prfcolor{\color{green}}
\def\thmcolor{\color{blue}}
\def\lemcolor{\color[rgb]{1,0,1}}
\def\obscolor{\color{black}}

\def\prf#1{{\bf \begingroup\prfcolor P\ifx#1\empty\else(\endgroup{\I #1}\begingroup\prfcolor)\fi:\endgroup}}
\def\thm#1{{\bf \begingroup\thmcolor T\ifx#1\empty\else(\endgroup{\I #1}\begingroup\thmcolor)\fi:\endgroup}}
\def\lem#1{{\bf \begingroup\lemcolor L\ifx#1\empty\else(\endgroup{\I #1}\begingroup\lemcolor)\fi:\endgroup}}
\def\obs#1{{\bf \begingroup\obscolor O\ifx#1\empty\else(\endgroup{\I #1}\begingroup\obscolor)\fi:\endgroup}}
\def\res#1{{\bf \begingroup\obscolor R\ifx#1\empty\else(\endgroup{\I #1}\begingroup\obscolor)\fi:\endgroup}}

\def\st{{\rm\ t.ž.\ }}
\def\iff{\leftrightarrow}
\def\then{\rightarrow}
\def\rng{{\rm Rng}}
% enumerability-related macros
\def\da{\downarrow}
\def\ua{\uparrow}
\def\daeq{\downarrow=}
\def\daneq{\downarrow\neq}

% ************  TEXT STARTS HERE **************************
\title{Vyčíslitelnost 1}
\section{Základní funkce}
$$\eqalign{ O(x) &= 0\cr
S(x) &= x+1\cr
I^j_n(x_1 \dots x_n) &= x_j\cr
S^m_n(f, g_{[1,m]}) &= h \st h(x_{[1,n]}) \simeq f(g_1(x_{[1,n]} \dots g_m(x_{[1,n]})) \cr
R_n(f,g) &= h \st \cases{h(0,x_{[2,n]}) = f(x_{[2,n]}) \cr h(i, x_{[2,n]}) = g(i, h(i, x_{[2,n]}), x_{[2,n]})\cr} \cr
M_n(f) &= h \st \cr
h(x_{[1,n]}) &\daeq z \iff \cases{f(x_{[1,n]}, z) \daeq 0 \cr \forall j < z: f(x_{[1,n]}) \daneq 0\cr} \cr}$$

\itemize{\dfn}
\:  $ČRF$ -- lze ji odvodit pomocí základních funkcí
\:  $ORF$ -- $ČRF$ a zároveň totální (všude definovaná)
\:  $PRF$ -- lze ji odvodit bez použití minimalizace
\endlist

\obs{} Zneužití značení v taháku: občas $ČRF$ třída (píši $f \in ČRF$), občas zastupuje slova
{\I částečně rekurzivní funkce} (píši) $f\ ČRF$ n. $f {\rm\ je\ } ČRF$. Mělo by být jasné
z kontextu.

\thm{} $ČRF \subset ORF \subset PRF$

\prf{}

\itemize\ibull
\:  $ČRF \&\neg ORF$ je prázdná funkce $f$: $g(x,y) = y+1, f = M_1(g)$.
\:  $ORF \&\neg PRF$ je Ackermannova funkce. Dá se naprogramovat, totální též, tedy je $ORF$. Pokud však definujeme
úrovně výpočtu pro každé $k$, tak se dokážeme pomocí $k$ for cyklů ($PRF$) dostat jen do hloubky $k$ a nikdy to
tedy nepokryjeme celé.
\endlist

\thm{} $TM \simeq ČRF$, co se výpočetní síly týče.

\prf{} 

\thm{Kleene} $\forall k \ge 1:$

\numlist\nalpha
\: $\exists ČRF \Psi_k(e, x_{[1,k]})$ univerzální pro $k$ proměnných.
\: Lze efektivně získat z funkce její $e$ a z $e$ funkci.
\: $\exists PRF_1 U \& \exists PRF_{k+2} T_k \st$ \hfil\break
$\Psi_k(e, x_{[1,k]}) \simeq U(\mu_y(T_k(e, x_{[1,k]},y)))$
\: $\exists PRF_{m+1} s_m$ prostá a rostoucí $\st$ \hfil\break
$\Psi_{m+n}(e,y_{[1,m]},x{[1,n]}) \simeq \Psi_m\left(s_m(e,y_{[1,m]}), x_{[1,n]}\right)$
\: Univerzální funkce je standardní.
\endlist

\thm{} Univerzální $PRF$ není sama $PRF$, univerzální $ORF$ není sama $ORF$. Univerzální $PRF$ je v $ORF$.

\prf{} První dvě části diagonalizací.

\section{Rekurzivní spočetnost}
\itemize{\dfn}
\: $\chi \equiv$ pravdivostní funkce výroku nebo charakteristická funkce mny.
\: $PRP/ORP$ (predikáty) $\equiv$ jeho $\chi$ je $PRF/ORF$.
\: $RSP$ predikát $\equiv$ obor konvergence $ČRF$.
\: $M$ mna je rekurzivní $\equiv \chi \in ORF$.
\: $M$ mna je rekurzivně spočetná ($RSM$) $\equiv M = dom(f), f \in ČRF$.
\endlist

\thm{Post} $M$ rekurzivní $\iff$ $M$ i $\overline{M}$ jsou $RSM$. $P$ je $ORP$ $\iff$ $P$ a $\neg P$ jsou $RSP$.

\prf{} Tam: Když je $\chi \in ORF$, můžeme rozhodovat o přítomnosti i nepřítomnosti
efektivně. Zpátky: Prostě pustíme oba výpočty souběžně, jeden se zastaví, neboť jsou
obě $RSM$. Pro výroky se to získá z věty o selektoru (níže).

\thm{Vlastnosti $\Psi$}

\numlist\ndotted
\: $\Psi_k(e,x_{[1,k]})\da, \Psi_k(x_{[1,k]})\da$ jsou $RSP$, ale ne $ORP/PRP$.
\: $\neg\left(\Psi_k(e, x_{[1,k]})\da\right), \neg\left(\Psi_k(x_{[1,k]})\da\right)$ nejsou $RSP$.
\: $\Psi_k$ nelze rozšířit do $ORF$.
\endlist

\prf{}

\thm{O selektoru} $\forall RSP_{k+1} Q \exists ČRF_k \varphi \st$
$$\eqalign{
\Phi(x_{[1,n]})\da &\iff \exists y: Q\left(x_{[1,n]}, y\right)\cr
\Phi(x_{[1,n]})\da &\then Q\left(x_{[1,n]}, \varphi(x_{[1,n]})\right)\cr}$$

\prf{}

\itemize{\res{}}
\: $\varphi \in ČRF \iff$ má rek. spočetný graf.
\: $\forall RSM$ je oborem hodnot nějaké $ČRF$.
\: $\forall f \in ČRF: \rng(f) \in RSM$.
\endlist

\section{Převoditelnost}

\section{Generování RSM}

\section{Matjasevičova věta}

\section{Věty o rekurzi}

\section{Produktivní a kreativní mny}

\section{Neoddělitelné mny}

\section{Gödelova věta}

\section{Reprezentovatelnost}

\bye
