\input cheatmac

\title{Graph theory}
{\bf Note}: Strange structure due to finals topics.
Will improve later.

\section{Graph coloring}

\thm{A $d$-degenerate graph can be coloured with $d$+1 colours.}

\prf{Greedy.}

\thm[Brooks]{A graph can be coloured using $\Delta$ colours if it's not an odd cycle or a complete graph.}

\prf{The choosability version of this follows from ERT theorem.}

\thm[Vizing]{A graph is edge-colorable using either $Δ$ or $Δ+1$ colours.}

\prf{Induction on $||G||$. For a coloring of $G - uv$, one colour is missing in $δ(u)$ and one in $δ(v)$. A maximal
graph on those two color classes has a path from $u$ to $v$, otherwise done.

Choose vertices $xy_0$. Find a maximal set of vertices of $δ(x)$ such
that the colour missing at $y_0$ is the colour of the edge $xy_1$, the
colour missing at $y_1$ is present on the edge $xy_2$ etc.

Take last $xy_k$ and its missing colour $β$. Create Kempe chain to some
$y_i$ to $x$. Switch to $y_i$, find Kempe chain again. Contradiction.

}

\thm[Thomassen]{Every planar graph is $5$-choosable.}

\prf{Every face bounded by a triangle, outer face a cycle. Suppose one vertex is coloured $1$, other $2$
and the rest on the outer face have $L(v)=3$ and the inside vertices $L(v)=5$. Then $G$ colourable.

Apply induction, look for chord, remove vertices if outer face chordless.
}

\subsection{Erdos-Rabin-Taylor}

\dfn{$G$ is a \term{Gallai block} $≡$ $G$ is $K_n$ or $C_{2k+1}$. } 

\dfn{$G$ is a Gallai tree $≡$ connected and (maximal) Gallai blocks form a tree.}

\obs{Graph connected, lists of degree-size, one bigger. Then $G$ is colorable.}

\res{1-connected graph with articulation can be coloured.}

\thm[Erdos-Rabin-Taylor]{$G$ connected. $G$ degree-choosable $\iff$ $G \neq $ Gallai tree.}

\lem[ERT lemma]{$G 2-$conn., $G$ not a Gallai tree block. Then $∃v ∈ G$ and
$v' \not ~ v'' ∈ δ(G)$ s.t. $G - v' - v''$ conn.}

\prf[ERT lemma]{ $G$ 3-connected: pick any non-adjacent vertices.

$G$ only $2$-conn.: Pick a $x,y$ cut such that the minimal component is minimal.
\itemize\ibull
\: $xy \in ||G||$: Pick $v,v'$ two neighbours of $y$. Use Menger's theorem to assert
victory.
\: $xy \notin ||G||$: Choose the same $v',v''$ neighbors of $y$. 
Define $Z_x$: vertices connected to $x$ after removing $v',v''$. Define $Z_y$ analogously.

Consider $C_0 \equiv$ the smallest $G - x -y$. Both $Z_x$ and $Z_y$ have empty intersection
with it, or we're done.

In the end, only $v'$ is connected to $x$ and $y$, apply induction on $H = G/v'$. Case analysis
based on whether or not is $H$ complete.
\endlist
}

\lem[Application of ERT Lemma]{$G$ 2-connected, not a Gallai block. Then it is deg-choosable.}

\prf{Assume there are different lists $L(u), L(v)$. We can find $u,v$ that are neighbors with
this property. Think of $u$ as root. Colour the rest of the graph greedily because $v$ suddenly
has more colours than neighbors.

So there are no such lists. All the vertices are of the same degree. If it's an odd cycle, we are done.
If not, use ERT lemma and find two vertices and use the greedy argument.
}

\prf[Erdos-Rabin-Taylor]{ $G$ connected. Proving stronger assertion: If $G$ is not connected,
it is a Gallai tree and each block has a ``designated'' list which it all shares.

By induction on block size. One block: okay. More blocks: cut off corner block,
proceed by induction.

} 

\res[Brooks]{If a graph $G$ is connected, it can either be coloured with lists of size $deg$ or
it is a $K_l$ or a $C_{2k+1}$.}


\lem[Kernel lemma]{ Let $G$ be a graph with associated lists for colouring. If $G$
has an orientation with $d^{+}(v) < |S_v|$ everywhere, and this orientation also
has a kernel for every induced subgraph, then $H$ can be coloured by $S_v$.
}

\prf{ Induction. Take a colour $α$, find all vertices that have it in their lists,
find their kernel. Colour kernel, remaining vertices still fulfill conditions, induce.
}

\thm[Galvin]{$χ'(G) = χ_L'(G)$ for bipartite graphs.}

\prf{We need to find an orientation for the line graph sasisfying the lemma. Take two partitions. Take $k$-coloring.
Assign orientation based on colouring, and based on where the two
edges meet ($X$ or $Y$). Prove condititons.  }

\opn[Hadwiger]{$k$-colorability implies a minor of $K_k$.}

Proven for $5$, $6$.

\opn[List Coloring Conjecture]{Edge choosability and edge colorability coincide.}

\section{Regular graphs}

\thm[Moore]{$G$ d-regular without $K_3, K_4$ and with exactly $n=d^2+1$ vertices: $d \in {2,3,7,17}$.
}

\obs{For a Moore graph and its adjacency matrix, $A^2 = J - A + (d-1)I$.}

\prf[Moore]{We know that $Sp(J) = \{n,0^{d-1}\}$. Because of observation, the eigenvalues also obey the
polynomial property. Therefore, we have $\lambda^2 + + \lambda - (d-1)1 = n$ or $0$.

For the main eigenvalue $d$, we get the condition $d^2 +1 = n$. For the
remaining conditions the right side is zero. Solving it as a quadratic
equation, we get solutions for $\lambda_1$ and $\lambda_2$ which are not $d$.
The discriminant will be $\sqrt{4d-3}$. Condition on it being rational or not,
and you get the remaining numbers.  }

\thm[Turan]{$∀r>1, n$ every graph without a $K^r$ subgraph and $ex(n,K^r)$ edges is a
Turan graph $T^{r-1}$.}

\prf{Among complete $k$-partite graphs, Turan graphs have the most edges. Among Turan
graphs, $T^{r-1}$ has the most edges. We need to prove that a graph with $ex(n,K^r)$ edges
is a complete multipartite graph.

If not, non-adjacency is not equivalence, and so find three conflicting vertices. Duplicating
some yields the contradiction.
}

\obs{$t_{r-1}(n) ≤ n^2/2 {r-2 \over r-1}$.}

\dfn{Density $d(A,B) = ||(A,B)|| / |A||B|$.}

\dfn{An {\it $ε$-regular pair} $(A,B)$ for given $ε$ has the property that every set $(X,Y)$
of size at least $εX, εY$ respectively has density $d(X,Y)$ $ε$-close to $d(A,B)$.
}

\dfn{An {\it $ε$-regular partition} for given $ε$ satisfies the following:
\itemize\ibull
\: $|V_0| ≤ ε|V|$.
\: $|V_i| = |V_j|$.
\: All but most $εk^2$ pairs are not $ε$-regular.
\endlist
}

\lem[Regularity lemma]{For every $ε$ and every $m ∃ M$ such that every graph of size $≥ m$
admits an $ε$-regular partition.}

\thm[Erdos-Stone]{For all $r≥2, s≥1, ε>0 ∃ n_0$ such that all graphs with more vertices
and at least $t_{r-1}(n) + ε n^2$ edges contain $K_r^s$ as a subgraph.}

\lem[Removal lemma]{ TBD.}

\section{Graph connectivity}

\thm[Menger]{$G$ graph. Then minimum number of vertices separating $A$ from $B$ in $G$
is equal to the maximum number of disjoint $A-B$ paths in $G$.}

\prf{Induction on $||G||$. If $G$ has no $k$ disjoint $A-B$ paths, contract one edge
$e = xy$. $G/e$ contains an $A-B$ separator $Y$ of less than $k$ vertices. One of them
must be the contracted $v_e$. Therefore $Y-v_e+x+y$ must be a separator in $G$ of exactly
$k$ vertices.

Consider now $G-e$. Every $A-X$ separator in $G-e$ is an $A-B$ separator in $G$, and so each such
separator has at least $k$ vertices. Therefore, there are $k$ disjoint $A-X$ paths and $k$
disjoint $B-X$ paths. These paths do not meet outside $X$, and can be extended to $A-B$ paths.
}

\thm[Mader]{

}

\lem{There is a function $h$ such that every graph of average degree $h(r)$ contains
$K^r$ as a topological minor.}

\prf{We show by induction on $m=r … {r \choose 2}$ that every graph with average degree
$≥ 2^m$ has a topological minor $X$ with $r$ vertices and $m$ edges.

Induction start can be done easily with a maximal cycle. Now, consider $d(G) ≥ 2^m$.
Assume $G$ connected. Find maximal set $U$ s.t. $d(G/U) ≥ 2^m$. Vertex with minimum degree
is acceptable, so there exists such a nonempty set, and $δ(U) ≠ ∅$.

Take neighborhood $δ(U)$. If there is any vertex with degree < $2^{m-1}$, we can add it.
Thus, minimum/average degree is $≥ 2^{m-1}$. Apply induction. Take topological minor
and connect two vertices through $U$, which is connected.
}

\thm{$k$-linked implies $f(k)$-connected.}

\prf{Prove it for $f(k) = h(3k) + 2k$. Find topological minor of $K^{3k}$. Use Menger
on its vertices and $2k$ vertices $S$ and $T$. Choose $k$ vertices in $K^{3k}$ that do not
have endvertices on the $2k$ disjoint paths by Menger. Connect the paths.
}

\section{Special properties of oriented graphs}

\section{Algebraic properties of graphs}

\subsection{Spectral theory}

\dfn{$\lambda$ is an {\it eigenvalue} $\equiv \exists x: Ax = \lambda x$. }

\obs{$Au - \lambda u = 0 \iff det |A - \lambda I| = 0$.}

\obs{$A$ has a base of eigenvectors $\iff$ $\exists X: X^*X=E$ and $X^*AX = I*(\lambda_1, \lambda_2, \dots)$.}

\res{$G$ graph, $A$ its adjacency matrix $\rightarrow$ there are $n$ different real eigenvalues of $A$.}

\res{$G$ graph $d$-regular $\rightarrow$ $\Lambda_{max} = d$. }

\obs{There exist cospectral graphs, for example $K_{1,4}$ and $C_4 + v$.}

\subsection{Flows and Tutte polynomial}

\section{Matching theory}

\thm[Hall]{Bipartite graph has PM iff $∀ A⊆X: δ(A) ≥ |A|$. }
\prf{Use Menger or mincut-maxflow on $G$. If there is not a perfect matching,
there exist $A ⊆ X$ and $B ⊆ Y$ such that $|A| + |B| < |X|$ there is no edge from $X ∖ A$ to $Y ∖ B$.
Then $δ(X ∖ A) ⊆ B$ and $|δ(X ∖ A)| ≤ |B| < |X| - |A| = |X ∖ A|$.}

\thm[Tutte]{A graph has a 1-factor iff $∀S: q(G ∖ S) ≤ |S|$, where $q$ counts the number of odd components.}

\prf{Choose $S_0$ maximal such that the inequality becomes an equality.
Prove then that even components have a PM, odd components without a
vertex have a PM and finally using Hall that $S_0$ to $\{C|C$ odd $\}$
has a PM.  }


Edmonds' Algorithm: Find an Edmonds forest, contract flowers (odd cycles).

\section{Ramsey theorey}

\dfn{$[X]^k$ is a set of all $k$-tuples of $X$.}

\thm[Infinite Ramsey]{Let $k,c$ be positive numbers, $X$ infinite. If we color $[X]^k$ by $c$ colors,
we find a monochromatic infinite subset.}

\prf{Prove it by induction on $k$, with $c$ fixed. Construct a series of sets $X_i$ and associated elements $x_i$ such that:
\itemize\ibull
\: $X_{i+1} = X_i - x_i.$
\: All $k$-sets $x_i ∪ Z$ with $Z$ from $X_{i+1}$ are monochromatic with colour that we associate with $x_i$.
\endlist

Just pick $x_i$ arbitrarily and note that if you associate the colour of $x_i$ to $k-1$-sets of $X_i - x_i$,
you can apply induction on $k$ and get condition 2.

Since $x_i$ is an infinite set, at least one colour occurs infinitely
many times, and this colour is the monochromatic subset.

}

\thm[Finite Ramsey]{
For every $k,c,r$ there exists $n≥k$ such that all $[n]^k$ colourings with $c$ colours contain a monochromatic
subset of size $r$.
}

\prf{If not, for all $n≥k$ there is a bad colouring. Use Konig's Infinity Lemma (compactness) and extend the bad
colourings into an infinite one. This contradicts infinite Ramsey.}


\thm[Erdos-Szekeres]{For $k≥2, l≥ 2$, every non-degenerate set of ${k+l-4 \choose k-2} +1$ points in the plane
contains a $k$-cup or a $l$-cap. Also for less points, a set avoiding
this exists.}

\prf{Induction. Suppose that there is a set of ${k+l-4 \choose k-2} + 1$ points an it has neither.
Choose $L$ the list of last points of $k-1$-cups. Then $X ∖ L$ has no $k-1$ cup or $l$ cap. Apply induction.
$L$ is therefore big enough and (by induction) contains a $l-1$ cap.
}

\thm[Van-der-Waarden]{Given $k,r$ there exists $n_0$ such that if we colour all sequences of length $≥ n_0$
with $k$ colours, we get a monochromatic arithmetic sequence of length $r$.}

\thm[Hales-Jewett]{$\forall a$ (cube size) and $r$ (color number) there exists $N = HJ(a,r)$ such
that colouring the cube $A^N$ with $r$ colours produces a monochromatic combinatorial line.}

\prf[VdW from HJ]{Use $f((x_1, x_2, \dots, x_n)) = \sum_i x_i$. Given colouring of $[ft]$, colour
element $\in K$ with the colour $c(f(x_1, x_2 \dots))$. Note that any combinatorial line is an
arithmetic sequence.
}

\goodbreak

\section{Infinite combinatorics}

\thm{Every connected graph contains a spanning tree.}

\prf{Apply Zorn's lemma on a set of finite inclusion-wise ordered spanning trees.}

\thm[König's Infinity Lemma]{Let $V_0, V_1, …, $ be an infinite sequence of non-empty disjoint
finite sets of vertices. $G$ graph of their union. Assume that every vertex in $V_i$ has a neighbor
in $V_{i-1}$. Then $G$ contains a ray that traverses all the sets.}

\prf{

}

\bigskip
Infinite Ramsey.

\bigskip


\section{Structural properties of set systems}

\thm[Sperner's Theorem]{
}

\bigskip

\thm[Dilworth]{

}

\bigskip

\thm[Sunflower lemma]{
}

\bigskip

\thm[Bollobas]{
}

\thm[Kruskal-Katona]{
}

\bye
