\input cheatmac

\title{Pravděpodobnostní metoda}

\section{Nástroje a odhady}

\thm{Union-bound}
$$P[\bigcup A_i] \le \sum_{i=0}^n P[A_i].$$
\thm{Čebyšev}
$$P[|X-E[X]| \ge \lambda\sigma] \le {1 \over \lambda^2}.$$
\prf{}
$$\sigma^2 = \var[X] = E[(X - E[X])^2] \ge \lambda^2 \sigma^2 P[|X-E[X]| \ge \lambda\sigma. $$
\thm{Markov}
$$P[X > \alpha E[X]] < {1 \over \alpha}.$$
\thm{Černov} Nechť $S_n$ je součtová náhodná proměnná pro $n$ mincí s hodnotami ${+1,-1}$. Pak:
$$P[S_n > a] < e^{-a^2 / 2n}.$$

\section{Základní použití}

\thm{Hrubý Ramsey} $\forall k \ge 3: R(k,k) > 2^{k/2 - 1}$.

\prf{} Vezmi $G(n,1/2)$, použij union-bound na jev \uv{$G$ obsahuje kliku nebo
nezávislou velikosti $k$}.

\dfn{}$m(k)$ bude označovat minimálni počet hran $k$-unif. hypergrafu, který není
$2$-obarvitelný. $m(2) = 3$ -- Fano.

\obs{} $m(3) \ge 7$.

\prf{} $6$ hran hypergrafu. Nejvýš $6$ vrcholů: rozeber. Aspoň $7$: dva nejsou spojené hranou, tak je spoj do jednoho vrcholu.

\thm{Odhad $m(k)$} $m(k) \ge 2^{k-1}$.

\prf{} Vezmi menší náhodný hypergraf. Zvol náhodné obarvení, spočti
pravděpodobnost že existuje jednobarevná (a tedy špatná) hrana.

\thm{} Pokud ${n \choose k} (1 - 2^{-k})^{n-k} <1$, tak existuje turnaj
s vlastností, že každých $k$ vrcholů má společného vst. souseda.

\prf{} Union bound.

\section{Náhodné permutace}

\dfn{}Systém množin $F$ {\I má průnik} pokud mají průnik všechny dvojice množin z $F$.

\thm{Slunečnicové lemma} Nosná množina $n \ge 2k$. Pokud má systém
$k-prvkových$ množin $F$ průnik, tak $|F| \le {n-1 \choose k-1}$.

\lem{} $X = Z_n$, zadefinuj $A_s = \{k,k+1,\dots s+k-1\}$ pro $s \in X$, $n$ buď
větší jak $2k$.  Pak $F$ s průnikem obsahuje nanejvýš $k$ z množin $A_s$.

\prf{} Když tam patří $A_i$, tak tam patří nanejvýš ty, co se s ní pronikají, a
ty se dají rozdělit ještě do exkluzivních dvojic.

\prf{Slunečnicové lemma} Budeme volit náhodné $s$ a náhodnou $\sigma$, co přepermutuje
$A_s$. Pak $P[\sigma(A_s) \in F] \le k/n$.

\section{Linearita střední hodnoty}

\thm{} Existuje turnaj, který má alespoň $n!/2^{n-1}$ Ham. cest.

\prf{} Rozděl jev na jednotlivé permutace. 

\thm{} Každý graf s $m$ hranami obsahuje bipartitní graf s $m/2$ hranami.

\prf{} Derandomizací nebo i deterministicky.

\fsection{Alterace}

\thm{Slabý Turán} $\alpha(G) \ge n/2d$.

\prf{} Vezmi náhodnou podmnožinu $S$, každý vrchol vlož s pravd. $p$.
$E[|S|] = np$. $E[||S||] = {ndp^2 \over 2}$. $|S| - ||S||$ odpovídá
tomu, že smažeme z každé hrany jeden vrchol. Zvolme $p$, ať
maximalizujeme $E[|S| - ||S||]$.

\thm{Další Ramsey} $\forall k,n: R(k,k) > n - {n \choose k} 2^{1 - {k \choose 2}}$.

\prf{} Vezmi náhodné obarvení grafu na $n$ vrcholech. Střední hodnota
počtu jednobarevných $k$-klik je ${n \choose k} 2^{1 - {k \choose
2}}$. Existuje obarvení, co to dosahuje, z každé kliky smaž jeden
vrchol.

\thm{Erdös} $\forall k,l \exists G \st \chi(G) > k \& g(G) > l$.

\prf{} Nastav $\epsilon = 1/2l, p = n^{\epsilon-1}$. Střední hodnota počtu cyklů
délky $l$ je $$E[X] \le \sum_{i=3}^l n^{\epsilon i} = o(n).$$
Zvol $n \st E[X] < n/4$. To nám dá $E[X] < 1/2$, použij Markova na $P[X \ge n/2] < 1/2$.

Teď počitej barevnost pomocí největší nezávislé. (Barevné třídy jsou
nezávislé.) $a$ buď $\lceil 3/p \ln n \rceil$, pak máme
$$P[\alpha \ge a] \le {n \choose a} (1-p)^{{a \choose 2}} \le n^a e^{-p{a \choose 2}} =
e^{(\ln n -p(a-1)/2)a}.$$

To jde do nuly, tedy zase máme $P[\alpha] < 1/2$. Smaž jeden vrchol ze všech krátkých cyklů.
Zbyde půlka vrcholů, barevnost odhadni velikostí $\alpha$.

\thm{} Existuje množina $S$ $n$ vrcholů v jednotkovém čtverci, kde
největší trojúhelník má obsah alepsoň $T(S) \ge 1/(100n^2)$.

\prf{} Zvol tři body $A,B,C$ uniformně z množiny všech bodů ve
čtverci. Nechť je $|AB| = x$. Chceme odhadnout $P[\mu(PQR) < c]$.
Nejdřív najdeme vzorec pro $P[b \le x \le b + \Delta b]$. Kdyby to
platilo, tak jsou $A$ a $B$ od sebe někde mezi dvěma kruhy, čili $P[b
\le x \le b + \Delta b] \le \pi(b+\Delta b)^2 - \pi b^2$. V limitě
$P[b \le x \le b + db] \le 2\pi b db$.

No a pokud vzdálenost je kolem $b$, pak výška musí být nejvýš $2c/b$,
což znamená, že $R$ je v pásu šířky $4c/b$ a šířky $\sqrt{2}$. Nyní
zintergruj možnosti podle $b$:

$$ \int_0^{\sqrt{2}}(2 \pi b)(4 \sqrt{2} c / b) db = 16 \pi c. $$ 

Zvolíme $c = 1/(100n^2)$ a spočítáme střední hodnotu počtu trojic,
které splňují, že jejich obsah je víc než $c$. Je to $E[T] \le {2n
\choose 3}(0.6n^{-2}) < n$. Alteruj.

\thm{Komlós, Pintz, Szemerédi} $T(S) = \Omega({\log n \over n^2})$.
 
\section{Dependent Random Choice}

\section{Rozptyl}

\thm{} $\forall m\ge 1: {2m \choose m} \ge 2^{2m}/(4\sqrt{m}+2)$.

\prf{} Vezmi náhodný součet $2m$ $0/1$-proměnných s $p=1/2$. $E[X] = m, \var[X] = m/2$. Čebyšev dává: $P[|X-m| < \sqrt{m}] \ge 1/2$. 
Pravděpodobnost, že $X$ nabyde přesně $m+k$ pro $|k| < \sqrt{m}$, 
je nejvýš ${2m \choose m-k}2^{-2m} \le {2m \choose m}2^{-2m}$.
Celkem tedy:

$$1/2 \le \sum_{|k| < \sqrt{m}} P[X = m+k] \le (2\sqrt{m} -1) {2m \choose m} 2^{-2m}.$$

\dfn{} $H$ graf je {\I balancovaný}, pokud jeho hustota je větší nebo
rovna všem jeho podgrafům.

\thm{} Pokud je $H$ balancovaný, $\rho$ jeho hustota, pak funkce $n^{-1/\rho}$ je jeho prahová funkce.

\section{Lovászovo lokální lemma}

\thm{Symetrické LLL} Buď $A_1, \dots, A_n \st P[A_i] \le p$ a všech\-ny vý\-stup\-ně v zá\-vi\-slostním
orientovaném grafu jsou nanejvýš $d$. Pokud $ep(d+1) \le 1$, pak 
$$P[\bigcup_{i=1}^n \overline{A_i}] > 0.$$

\prf{} Buď $A_i$, nastav k nim reálné 0-1-proměnné $x_i$. Pokud platí
$$P[A_i] \ge x_i \prod_{(i,j) \in E} (1-x_j), $$
Pak
$$P[\bigcap_i \overline{A_i}] \ge \prod_i (1-x_i) > 0.$$

\prf{} Je třeba počítat $A_i$ (to, co nechceme, aby se stalo) za předpokladu, že
už se nějaka podmnožina $S$ jevů nestala. Indukcí dokazujeme:

$$ P[A_i|\bigcap_{j \in S} \overline{A_j}] \le x_i.$$

%TODO: Doplnit.

\thm{Vážené LLL} $A_i$ jako vždy, $0 \le p \le 1/4$, reálná čísla $t_i$.
S kladnou pravděpodobností se nic zlého nestane, pokud:
\itemize\ibull
\: $P[A_i] \le p^{t_i}$,
\: $\sum_{(i,j) \in E} (2p)^t_j \le t_j /2$.
\endlist

\dfn{} Obarvení je {\I silně hvězdicové} ($\chi_h$), pokud indukovaný graf každých dvou barev
je hvězda (plus izolované vrcholy).

\thm{} Pro $G$ s $\Delta$ a $m$ hranami, $\chi_h(G) \le 14 \sqrt{\Delta m}$.

\prf{} 

\section{Černov a spol.}

\bye
