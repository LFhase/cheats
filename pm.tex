\input cheatmac
\usepackage[czech]{babel}
\begin{document}
\begin{multicols}{3}

\title{Pravděpodobnostní metoda}


\fsection{Nástroje a odhady}
\thm[Union-bound]{
$P[⋃ A_i] ≤ ∑_{i=0}^n P[A_i]$.
}

\thm[Markov]{ $X$ nezáporná, $a > 0$:
$P[X > α E[X]] < {1 \over α}.$
}

\thm[Čebyšev]{
$P[|X-E[X]| ≥ λσ] ≤ {1 \over λ^2}.$
}

\prf{
$\sigma^2 = \var[X] = E[(X - E[X])^2] \ge \lambda^2 \sigma^2 P[|X-E[X]|] \ge \lambda\sigma. $
}

\thm[Černov]{
Nechť $S_n$ je součtová náhodná proměnná pro $n$ mincí s hodnotami ${+1,-1}$. Pak:
$P[S_n > a] < e^{-a^2 / 2n}.$
}

\prf{
Vytvoříme náhodnou proměnnou $Y = e^{uX}$.

$E[Y] = \prod_{i=1}^n E[e^{uX_i}] = {e^u + e^{-u} \over 2}^n \le e^{nu^2/2}$. Použitím Markova pak
$P[X \ge t] = P[Y \ge e^{ut}] \le E[Y] / e^{ut} \le e^{nu^2/2 - ut}$. Zvol $u =
t/n$.
}

\fsection{Základní použití}

\thm[Hrubý Ramsey]{
$\forall k \ge 3: R(k,k) > 2^{k/2 - 1}$.
}

\prf{
Vezmi $G(n,1/2)$, použij union-bound na jev \uv{$G$ obsahuje kliku nebo
nezávislou velikosti $k$}.
}

\dfn{
$m(k)$ bude označovat minimálni počet hran $k$-unif. hypergrafu, který není
$2$-obarvitelný. $m(2) = 3$ -- Fano.
}

\obs{
$m(3) \ge 7$.
}

\prf{ $6$ hran hypergrafu. Nejvýš $6$ vrcholů: rozeber. Aspoň $7$: dva nejsou
spojené hranou, tak je spoj do jednoho vrcholu.}

\thm[Odhad $m(k)$]{
$m(k) \ge 2^{k-1}$.
}

\prf{
Vezmi menší náhodný hypergraf. Zvol náhodné obarvení, spočti
pravděpodobnost že existuje jednobarevná (a tedy špatná) hrana.
}

\thm{
Pokud ${n \choose k} (1 - 2^{-k})^{n-k} <1$, tak existuje turnaj
s vlastností, že každých $k$ vrcholů má společného vst. souseda.
}

\prf{
Union bound.
}

\fsection{Náhodné permutace}

\dfn{
Systém množin $F$ \textit{má průnik} pokud mají průnik všechny dvojice množin z $F$.
}

\thm[Slunečnicové lemma]{
Nosná množina $n \ge 2k$. Pokud má systém
$k$-prvkových množin $F$ průnik, tak $|F| \le {n-1 \choose k-1}$.
}

\lem[Lemma pro lemma]{
$X = Z_n$, zadefinuj $A_s = \{s,s+1,\dots s+k-1\}$ pro $s \in X$, $n$ buď
větší jak $2k$.  Pak $F$ s průnikem obsahuje nanejvýš $k$ z množin $A_s$.
}

\prf{
Když tam patří jedna $A_i$, tak tam patří nanejvýš ty, co se s ní pronikají ($2k-2$), a
ty se dají rozdělit ještě do disjunktních dvojic.
}

\prf[Slunečnicové lemma]{
Budeme volit náhodné $s$ a náhodnou $\sigma$, co
přepermutuje $A_s$. Díky lemmatu je nejvýš $k$ obraz $\sigma(A_s)$ uvnitř $F$,
čili $P[\sigma(A_s) \in F] \le k/n$.  Každá $k$-tice má ale stejně permutací,
ve kterých je za sebou, čili taky $P[\sigma(A_s) \in F] = |F|/{n \choose k}$.
Počítej dvěma způsoby.
}

\dfn{
Definujme $n(k,l)$ jako maximální $n \st \exists \mls A n \andamp \mls B n \st$

\begin{enumerate}
\item $|A_i| = k, |B_i| = l,$
\item $A_i \cap B_i = \emptyset,$
\item $i \neq j \then A_i \cap B_j \neq \emptyset.$
\end{enumerate}
}

\thm[Bollobásovo lemma o průnicích]{
$n(k,l) = {k + l \choose k}$.
}

\prf{
Náhodně přepermutuj prvky. Zadefinuj jev: $U_i \equiv$ všechny prvky
$A_i$ přechází všechno z $B_i$.  $P[U_i] = 1/{k +l \choose k}$. $U_i$ a $U_j$
nemohou nastat zároveň, čili $1 \ge P[\bigcup U_i]  = \sum_i P[U_i] = n {k+l \choose k}$.
}

\res{ Spernerova věta.}

\fsection{Linearita střední hodnoty}

\thm{
Existuje turnaj, který má alespoň $n!/2^{n-1}$ Ham. cest.
}

\prf{
Rozděl jev na jednotlivé permutace. 
}

\thm{
Každý graf s $m$ hranami obsahuje bipartitní graf s $m/2$ hranami.
}

\prf{
Derandomizací nebo i deterministicky.
}

\fsection{Alterace}

\thm[Slabý Turán]{
$\alpha(G) \ge n/2d$.
}

\prf{
Vezmi náhodnou podmnožinu $S$, každý vrchol vlož s pravd. $p$.
$E[|S|] = np$. $E[||S||] = {ndp^2 \over 2}$. $|S| - ||S||$ odpovídá
tomu, že smažeme z každé hrany jeden vrchol. Zvolme $p$, ať
maximalizujeme $E[|S| - ||S||]$.
}

\thm[Další Ramsey]{
$\forall k,n: R(k,k) > n - {n \choose k} 2^{1 - {k \choose 2}}$.
}

\prf{
Vezmi náhodné obarvení grafu na $n$ vrcholech. Střední hodnota
počtu jednobarevných $k$-klik je ${n \choose k} 2^{1 - {k \choose
2}}$. Existuje obarvení, co to dosahuje, z každé kliky smaž jeden
vrchol.
}

\thm[Erdös, obvod neurčuje barevnost]{
$\forall k,l \exists G \st \chi(G) > k \andamp g(G) > l$.
}

\prf{
Nastav $\varepsilon = 1/2l, p = n^{\varepsilon-1}$. Střední hodnota počtu cyklů
délky $l$ je \[E[X] \le \sum_{i=3}^l n^{\varepsilon i} = o(n).\]
Zvol $n \st E[X] < n/4$. To nám dá $E[X] < 1/2$, použij Markova na $P[X \ge n/2] < 1/2$.

Teď počitej barevnost pomocí největší nezávislé. (Barevné třídy jsou
nezávislé.) $a$ buď $\lceil 3/p \ln n \rceil$, pak máme
\[P[\alpha \ge a] \le {n \choose a} (1-p)^{{a \choose 2}} \le n^a e^{-p{a \choose 2}} =
e^{(\ln n -p(a-1)/2)a}.\]

To jde do nuly, tedy zase máme $P[\alpha] < 1/2$. Smaž jeden vrchol ze všech
krátkých cyklů.  Zbyde půlka vrcholů, barevnost odhadni velikostí $\alpha$.
}

\thm{
Existuje množina $S$ $n$ vrcholů v jednotkovém čtverci, kde
největší trojúhelník má obsah alepsoň $T(S) \ge 1/(100n^2)$.
}

\prf{
Zvol tři body $A,B,C$ uniformně z množiny všech bodů ve
čtverci. Nechť je $|AB| = x$. Chceme odhadnout $P[\mu(PQR) < c]$.
Nejdřív najdeme vzorec pro $P[b \le x \le b + \Delta b]$. Kdyby to
platilo, tak jsou $A$ a $B$ od sebe někde mezi dvěma kruhy, čili $P[b
\le x \le b + \Delta b] \le \pi(b+\Delta b)^2 - \pi b^2$. V limitě
$P[b \le x \le b + db] \le 2\pi b db$.

No a pokud vzdálenost je kolem $b$, pak výška musí být nejvýš $2c/b$,
což znamená, že $R$ je v pásu šířky $4c/b$ a šířky $\sqrt{2}$. Nyní
zintegrujeme možnosti podle $b$:

\[ \int_0^{\sqrt{2}}(2 \pi b)(4 \sqrt{2} c / b) db = 16 \pi c. \] 

Zvolíme $c = 1/(100n^2)$ a spočítáme střední hodnotu počtu trojic,
které splňují, že jejich obsah je víc než $c$. Je to $E[T] \le {2n
\choose 3}(0.6n^{-2}) < n$. Alterujeme.
}

\thm[Komlós, Pintz, Szemerédi]{
$T(S) = \Omega({\log n \over n^2})$.
}
 
\fsection{Dependent Random Choice}

\lem[Základní DRC]{
$G$ graf, $|G| = n, d$ prům. stupeň. Pokud existuje
přirozené číslo $t$ takové, že ${d^t \over n^{t-1}} - {n \choose r} {m \over
n}^t \ge a$, pak $G$ obsahuje podmnožinu vcholů $|U|=a \st \forall r$ vrcholů z
ní má alespoň $m$ společných sousedů (značíme $|N_s(R)| = m$.
}

\prf{
Zvol $t$ vrcholů $T$ náhodně s opakováním. $A = N_s(T)$, $X = |A|$. Použijeme
konvexitu $z^t$ pro nerovnost:
$E[X] = \sum_{v \in V(G)} \left({ |N_s(v)| \over n}^t\right) = n^{-t} \sum \left(|N_s(v)|^t\right) =$ $n^{-t} n (\sum |N_s(v)|/n)^t
= {d^t \over n^{t-1}}.$

$Y$ buď počet špatných podmnožin $A$, tj. s velikostí $r$ ale s méně jak $m$ společnými
sousedy. $P[S \subseteq A] = (|N(S)|/n)^t$, a tedy $E[Y] < {n \choose r} \left(m/n\right)^t$.

Alteruj,  $E[X-Y] \ge a$, smaž špatné vrcholy.
}

\lem{
Pokud máme podmnožinu $U \subseteq V(G), |U|=a$ takovou, že každá $r$ prvková podmnožina
má alespoň $a+b$ společnych sousedů, tak sem umíme vnořit $H$ bipartitní s partitami velikosti
$a,b$, kde jedna z nich má $\Delta = r$.
}

\prf{
Indukcí, společných sousedů je $a+b$ a to stačí pro vnořování vrcholů s max. stupněm
$b$ (společní sousedé mohou být opět v $U$, to nevadí).
}

\thm{
$H$ bipartitní graf s $\Delta = r$ v jedné partitě $\then$ $ex(n,H) \le cn^{2-1/r}$.
}

\prf{
Vol $m = a+b$, $t = r$, $c = \max\left(a^{1/r},{3a+b \over r}\right)$, $||G|| > ex$.
Pak $d \ge 2cn^{1-1/r}$, odhadni $r!$ pomocí $(r/e)^r$ a konečně:
\[ (2c)^r - {n^r \over r!} {(a+b)^r \over n}^r \ge c^r \ge a. \]

Pro celkové vítězství aplikujeme předchozí lemma.
}

\fsection{Rozptyl}

\thm{
$\forall m\ge 1: {2m \choose m} \ge 2^{2m}/(4\sqrt{m}+2)$.
}

\prf{
Vezmi náhodný součet $2m$ $0/1$-proměnných s $p=1/2$. $E[X] = m, \var[X] = m/2$. Čebyšev dává: $P[|X-m| < \sqrt{m}] \ge 1/2$. 
Pravděpodobnost, že $X$ nabyde přesně $m+k$ pro $|k| < \sqrt{m}$, 
je nejvýš ${2m \choose m-k}2^{-2m} \le {2m \choose m}2^{-2m}$.
Celkem tedy:
\[1/2 \le \sum_{|k| < \sqrt{m}} P[X = m+k] \le (2\sqrt{m} -1) {2m \choose m} 2^{-2m}.\]
}

\lem[O malém rozptylu.]{
 Nechť $X_i$ je posloupnost proměnných, pro které platí $\lim_{n \rightarrow \infty} {\var(X_n) \over E[X_n]^2} = 0$.
Pak $\lim_{n \rightarrow \infty} P[X_n > 0] = 1.$
}

\lem{
Pokud $k(n)$ je funkce splňující
$ \lim_{n \rightarrow \infty} {n \choose k(n)} 2^{-{k(n) \choose 2}} = \infty, $
pak platí:
$ \lim_{n \rightarrow \infty} P[\Omega(G(n,1/2)) > k(n)] = 1.$
}

\prf{
V podstatě dokazujeme pro $k(n) = (2-\varepsilon) \log n$. $E(n,k) \equiv$
střední počet $k$-klik v $G(n,1/2)$.  $k(n) < 2 \log n$, protože $E(n,2 \log n)
\rightarrow 0$. $k(n) > 3/2 \log n$, protože\hfil\break $\log E(n,3/2 \log n) = 3/2 \log^2
n - 9/8 \log^2 n \rightarrow \infty$.

Budeme chtít použit lemma o malém rozptylu. $X$ ($X_n$) buď proměnná počítající
počet $k(n)$-klik. Její rozptyl je součet kovariancí pro každou $k(n)$-tici
vrcholů zvlášť, navíc to ještě sečteme podle velikostí průniků $t$,  $C(t)$
bude daný součet pro průniky velikosti $t$.

Kovarianci dvou $k$-tic $S$,$T$ mohu odhadnout pomocí $E[X_S X_T] \le 2^{{t
\choose 2} - 2{k \choose 2}}.$ $C(t)$ pak kombinatoricky jako \[C(t) \le {n
\choose k}{k \choose t}{n - k \choose k - t} E[X_S X_T].\]

Budeme nyní chtít ulimitit $\sum_{t \ge 2}^k {C(t) \over E[X]^2 }$ do nuly,
trikem ji rozložíme na dvě sumy: do $k/2$ a nad $k/2$.

První část půjde do $0$, pokud $k < 2 \log n$. Výpočet:

$ {C(t) \over E[X]^2} \le {{k \choose t} {n -k \choose k-t} \over {n \choose k}} 2^{t \choose 2} \le 
k^2t (n^{\underline{t}} t!)^{-1} 2^{t^2/2} \le (k^2 2^{-k/2}2^{t/2})^{t}.$

Sumu pak omez jako geometrickou řadu. Druhá část počítá jen s jednou mocninou
$E[X]$. Použijeme, že $3/2 \log n \le k$.
\begin{align*}
{C(t) \over E[X]} &\le {k \choose k-t} {n \choose k-t} 2^{{t \choose 2} - {k \choose 2}} ≤ (kn2^{-(k+t-1)/2})^{k-t}\\
                  &\le (2^{\log k + 2k/3 - 3k/4})^{k-t}.
\end{align*}
Opět omez geometrickou řadou a použij fakt, že $E[X]$ jde do nekonečna, čili to
platí i pro druhé mocniny.
}

\dfn{
$H$ graf je \textit{balancovaný}, pokud jeho hustota je větší nebo
rovna všem jeho podgrafům.
}

\thm{
Pokud je $H$ balancovaný, $\rho$ jeho hustota, pak funkce $n^{-1/\rho}$ je jeho prahová funkce.
}

\dfn{
Množina $R \subseteq [n]$ bude mít \textit{různé sumy}, pokud její všechny
sumy jsou různé. $f(n)$ je maximální velikost takové $R$.
}

\thm[Menší zlepšení rozdílných sum]{
$f(n) \le \log_2 n + (1/2) \log_2 \log_2 n + O(1)$.
}

\prf{Máme zadané $\mls r k $, vezmi náhodnou sumu (každé číslo vezmi s pravd.
1/2). Spočítej střední hodnotu, rozptyl. (Indikátory jsou nezávislé.) Čebyšev
nám dává: $P[|X-\mu| > \lambda n \sqrt{k}/2] \le \lambda^{-2}$, čili i $1 -
\lambda^{-2} \le P[|X-\mu| \le \lambda n \sqrt{k}/2]$. Všimneme si dále, že
jedné konkrétní hodnoty nabude $X$ s pravděpodobností $0$ n. $2^-k$, čili
$P[\ldots] < 2^{-k}(\lambda n \sqrt{k} + 1)$. Spojíme nerovnosti a upravíme.
}

\section{Lovászovo lokální lemma}

\thm[Symetrické LLL]{ Buď $A_1, \dots, A_n \st P[A_i] \le p$ a všech\-ny vý\-stup\-ně v zá\-vi\-slostním
orientovaném grafu jsou nanejvýš $d$. Pokud $ep(d+1) \le 1$, pak 
\[P[\bigcup_{i=1}^n \overline{A_i}] > 0.\]
}

\thm[Asymetrické LLL]{ Buď $A_i$, nastav k nim reálné 0-1-proměnné $x_i$. Pokud platí
\[P[A_i] \ge x_i \prod_{(i,j) \in E} (1-x_j), \text{pak} \]
\[P[\bigcap_i \overline{A_i}] \ge \prod_i (1-x_i) > 0.\]
}

\prf{Je třeba počítat $A_i$ (to, co nechceme, aby se stalo) za předpokladu, že
už se nějaka podmnožina $S$ jevů nestala. Indukcí podle $S$ dokazujeme: $ P[A_i|\bigcap_{j \in S} \overline{A_j}] \le x_i.$

Základ indukce je hned z předpokladů. Dále rozdělíme $S$ na zajímavé $(S_1)$ a nezávislé na $A_i$ $(S_2)$.
Rozepíšeme:

\[P[A_i | \bigcap_{j \in S} \overline{A_j}] =
{P[A_i \cap \bigcap_{j \in S_1} \overline{A_j} | \bigcap_{l \in S_2} \overline{A_l}] \over
 P[\bigcap_{j \in S_1} \overline{A_j} | \bigcap_{l \in S_2} \overline{A_l}]}.
\]

Nahoře můžeme odhadnout pomocí $P[A_i | \dots]$, což dá $x_i$ z předpokladů věty,
neboť $S_2$ byly nezávislé na $A_i$. Dole použijeme indukci: rozdělíme jmenovatel
na součin nezávislých jevů, které všechny mají menší $S$:
$P[\bigcap_{j \in S_1} \overline{A_j} | \bigcap_{l \in S_2} \overline{A_l}] = 
P[A_{j_1} | \bigcap_{l \in S_2} \overline{A_l}] \cdot P[A_{j_2} | \overline{A_{j_1}} \cap \bigcap_{l \in S_2} \overline{A_l}] \cdots
\le (1-x_{j_1})(1-x_{j_2}) \cdots (1-x_{j_r}).$

}

\prf[Symetrické LLL]{Nastav $x_i$ na $1/(d+1)$. Pak $x_i \prod_{(i,j) \in E} (1-x_j) = (d+1)^{-1} \left( 1 - 1/(d+1)\right)^{d} \le (e(d+1))^{-1}
\ge p$ a můžeme použít asymetrické lemma.
}

\thm[Barvení hypergrafů]{$H$ hypergraf, kde každá hrana má alespoň $k$ vrcholů
a proniká nanejvýš $d$ hran. Pokud $e(d+1) \le 2^{k-1}$, tak $H$ je 2-obarvitelné.
}

\prf{Použij symetrickou verzi LLL na jev, že hrana je jednobarevná.}

\thm[Cykly dlouhé násobek]{$D$ digraf, minimální výstupeň $\delta$, maximální vstupeň
$\Delta$. Pak platí-li $k \le {\delta \over 1 + \ln(1+\delta\Delta)}$, tak $D$ obsahuje
orientovaný cyklus dělitelný $k$.}

\prf{Přiřaď vrcholům čísla a jev $A_v$ bude jev, kdy $\exists z \in N_+(v) \st$ $k(z) = k(v) +1 \mod k$.
$A_v$ pak nezávisí na barvě $v$, jen na barvách $N_+(v)$, čili závisí jen na takových $A_w$, kteří jsou
buď přímo následníci $v$ nebo mají s $v$ společného následníka. Těch je celkem $\delta \Delta$, čili
$ep(d+1) \le e(1-1/k)^\delta (\delta \Delta + 1) \le 1$.
}

\dfn{Obarvěme reálná čísla $k$ barvami. $S \subseteq R$ je \textit{duhová} $\equiv c(S) = [k]$.}

\thm{$\forall k \exists m \st \forall S \subseteq R, |S|=m$, umíme nabarvit
reálná čísla $k$ barvami tak, aby jakékoli posunutí $m$ bylo duhové.}

\prf{Konečně mnoho posunutí: jedno posunutí závisí na méně než $m^2$ jiných, 
pravděpodobnost neduhovosti je $k(1-1/k)^m$. Zvol $m$ správně.

Nekonečně mnoho posunutí: kompaktnost.
}

% \thm{Vážené LLL} $A_i$ jako vždy, $0 \le p \le 1/4$, reálná čísla $t_i$.
%S kladnou pravděpodobností se nic zlého nestane, pokud:
%\itemize\ibull
%\: $P[A_i] \le p^{t_i}$,
%\: $\sum_{(i,j) \in E} (2p)^t_j \le t_j /2$.
%\endlist

%\dfn{} Obarvení je \textit{silně hvězdicové} ($\chi_h$), pokud indukovaný graf každých dvou barev
%je hvězda (plus izolované vrcholy).

%\thm{} Pro $G$ s $\Delta$ a $m$ hranami, $\chi_h(G) \le 14 \sqrt{\Delta m}$.

%\prf{} 

\fsection{Černov a spol.}

\obs[Černov pro $\{0,1\}$ nez. p.]{$P[X \ge E[X] + t] \le e^{-2t^2/n}$ a
symetricky pod střední hodnotou.}

\obs[Černov pro $(0,1)$ nez. p.]{$P[X \ge E[X] + t] < \exp({-t^2 \over
2(\sigma^2 + t/3)})$ a symetricky pod střední hodnotou.
}

\dfn[$k$-pakování]{Máme rodinu $F, |F|=m$ podmnožin z $[n]$, hledáme max. podrodinu, že
každý element je nejvýše $k$-krát obsažen.} 

$k$-pakování můžeme reprezentovat jako matici $n \times m$ s hodnotami $0/1$,
hledáme největší podmnožinu sloupců, co se nasčítají na nejvýš $k$ v každé
složce.

Vyřeš lineárně, zaokrouhluj podle $j$-té složky optima ($x^{*}_j$) na $1$ nebo $0$.
$E[y] = OPT^{*}, E[(Ay)_i] \le k$, ale to nemusí být přípustné -- nastavíme $y_j = 1$
s pravd. $(1 - \varepsilon/2) x^{*}_j$. 

\thm{$\varepsilon \in [0,1]$, $k \ge {10 \over \varepsilon} \ln(2n+2)$. Pak s pravděpodobností
$> 1/2$ je $y$ přípustné s alespoň $(1-\varepsilon) OPT^*$ množinami.}

\prf{
Nastav $X$ součet v $j$-té složce matice. $OPT* > k$, $E[X] = (1
\varepsilon/2) OPT^*$, $\var[X] \le E[X]$ jako obvykle.

Pak $P[X < (1 - \varepsilon/2)OPT^*] \le \exp(-{\varepsilon^2 \over 10} OPT^*)
\le \exp(-{\varepsilon^2 \over 10}k \le {1 \over 2n+2}$.

Druhou stranu uděláme totožně.
}
\end{multicols}
\end{document}