\pdfoutput=1
\pdfpkresolution=600

\input ucwmac.tex
\input ucw-ofs.tex
\input color

%**start of header
\newcount\columnsperpage

% This file can be printed with 1, 2, or 3 columns per page (see below).
% [For 2 or 3 columns, you'll need 6 and 8 point fonts.]
% Specify how many you want here.  Nothing else needs to be changed.

\columnsperpage=3

% Smaller (97%) pdf file with horizontal offset 1.5in
% dvipdfm -l -m 0.97 -x 1.5in -o TeXRefCard.v1.5.pdf TeXRefCard.v1.5.dvi

% There are a couple extra sections included at the end of the document
% (after the \bye) that didn't fit into six columns.
% It would be nice to have additional sections covering:
% \hrules and \vrules, Registers
% \input and \output files (including \read, \write, \message)

% This reference card is distributed in the hope that it will be useful,
% but WITHOUT ANY WARRANTY; without even the implied warranty of
% MERCHANTABILITY or FITNESS FOR A PARTICULAR PURPOSE.

% This file is intended to be processed by plain TeX (TeX82).
%
% The final reference card has six columns, three on each side.
% This file can be used to produce it in any of three ways:
% 1 column per page
%    produces six separate pages, each of which needs to be reduced to 80%.
%    This gives the best resolution.
% 2 columns per page
%    produces three already-reduced pages.
%    You will still need to cut and paste.
% 3 columns per page
%    produces two pages which must be printed sideways to make a
%    ready-to-use 8.5 x 11 inch reference card.
%    For this you need a dvi device driver that can print sideways.
% Which mode to use is controlled by setting \columnsperpage above.
%

\def\versionnumber{1.5}  % Version of this reference card
\def\year{2007}
\def\month{January}
\def\version{\month\ \year\ v\versionnumber}


\def\shortcopyrightnotice{
}

\def\copyrightnotice{
}

% make \bye not \outer so that the \def\bye in the \else clause below
% can be scanned without complaint.
\def\bye{\par\vfill\supereject\end}

\newdimen\intercolumnskip
\newbox\columna
\newbox\columnb

\def\ncolumns{\the\columnsperpage}

\message{[\ncolumns\space
   column\if 1\ncolumns\else s\fi\space per page]}

\def\scaledmag#1{ scaled \magstep #1}

% This multi-way format was designed by Stephen Gildea
% October 1986.

% Modified for Czech fonts.
\if 1\ncolumns
   \hsize 4in
   \vsize 10in
   \voffset -.7in
   \font\titlefont=\fontname\tenbf \scaledmag3
   \font\headingfont=\fontname\tenbf \scaledmag2
   \font\smallfont=\fontname\sevenrm
   \font\smallsy=\fontname\sevensy

   \footline{\hss\folio}
   \def\makefootline{\baselineskip10pt\hsize6.5in\line{\the\footline}}
\else
   \hsize 3.2in
   \vsize 7.95in
   \hoffset -.75in
   \voffset -.745in
   \font\titlefont=csbx10 \scaledmag2
   \font\headingfont=csbx10 \scaledmag1
   \font\smallfont=csr6
   \font\smallsy=cssy6
   \font\eightrm=csr8
   \font\eighti=csmi8
   \font\eightsy=cssy8
   \font\eightbf=csbx8
   \font\eighttt=cstt8
   \font\eightit=csti8
   \font\eightsl=cssl8
   \textfont0=\eightrm
   \textfont1=\eighti
   \textfont2=\eightsy
   \def\rm{\eightrm}
   \def\bf{\eightbf}
   \def\tt{\eighttt}
   \def\it{\eightit}
   \def\sl{\eightsl}
   \normalbaselineskip=.8\normalbaselineskip
   \normallineskip=.8\normallineskip
   \normallineskiplimit=.8\normallineskiplimit
   \normalbaselines\rm          %make definitions take effect

   \if 2\ncolumns
     \let\maxcolumn=b
     \footline{\hss\rm\folio\hss}
     \def\makefootline{\vskip 2in \hsize=6.86in\line{\the\footline}}
   \else \if 3\ncolumns
     \let\maxcolumn=c
     \nopagenumbers
   \else
     \errhelp{You must set \columnsperpage equal to 1, 2, or 3.}
     \errmessage{Illegal number of columns per page}
   \fi\fi

   \intercolumnskip=.46in
   \def\abc{a}
   \output={%
       % This next line is useful when designing the layout.
       %\immediate\write16{Column \folio\abc\space starts with \firstmark}
       \if \maxcolumn\abc \multicolumnformat \global\def\abc{a}
       \else\if a\abc
        \global\setbox\columna\columnbox \global\def\abc{b}
         %% in case we never use \columnb (two-column mode)
         \global\setbox\columnb\hbox to -\intercolumnskip{}
       \else
        \global\setbox\columnb\columnbox \global\def\abc{c}\fi\fi}
   \def\multicolumnformat{\shipout\vbox{\makeheadline
       \hbox{\box\columna\hskip\intercolumnskip
         \box\columnb\hskip\intercolumnskip\columnbox}
       \makefootline}\advancepageno}
   \def\columnbox{\leftline{\pagebody}}

   \def\bye{\par\vfill\supereject
     \if a\abc \else\null\vfill\eject\fi
     \if a\abc \else\null\vfill\eject\fi
     \end}
\fi

% ***** Verbatim typesetting *****

% Verbatim typesetting is done by
%    \verbatim"stuff to verbatim typeset"
% Any character can be used in place of ".
% E.g. \verbatim?stuff? or \verbatim|stuff|.  Cf. TeXbook pp.380-382

\def\uncatcodespecials{\def\do##1{\catcode`##1=12}\dospecials}
\def\setupverbatim{\tt%
\def\par{\leavevmode\endgraf}\catcode`\`=\active%
\obeylines\uncatcodespecials\obeyspaces}
\def\verbatim{\begingroup\setupverbatim\doverbatim}
\def\doverbatim#1{\def\next##1#1{##1\endgroup}\next}

\def\\{\verbatim}
\def\ds{\displaystyle}
\def\SPC{\quad} % space between symbol and command

\parindent 0pt
\parskip 1ex plus .5ex minus .5ex

\def\small{\smallfont\textfont2=\smallsy\baselineskip=.8\baselineskip}

\outer\def\newcolumn{\vfill\eject}

\outer\def\title#1{{\titlefont\centerline{#1}}\vskip 1ex plus .5ex minus.5ex}

%\outer\def\section#1{\par\filbreak
%  \vskip 1ex plus 2ex minus 2ex {\headingfont #1}\mark{#1}%
%  \vskip 1ex plus 1ex minus .5ex}
\outer\def\section#1{\par\filbreak
   \vskip .75ex plus 1ex minus 2ex {\headingfont #1}\mark{#1}%
   \vskip .5ex plus .5ex minus .5ex}
\outer\def\subsection#1{\par\filbreak
   \vskip .75ex plus 1ex minus 2ex {\bf #1}\mark{#1}%
   \vskip .5ex plus .5ex minus .5ex}

\def\paralign{\vskip\parskip\halign}

%\def\<#1>{$\langle${\rm #1}$\rangle$}

\def\begintext{\par\leavevmode\begingroup\parskip0pt\rm}
\def\endtext{\endgroup}

% smaller indenting of itemize lists for a compact format
\itemindent=0.15in
\itemnarrow=0in

% math-related definitions (compressed than usual):

\def\dfn{
{\color{red} \bf D:}
}

\def\prfcolor{\color{green}}
\def\thmcolor{\color{blue}}
\def\lemcolor{\color[rgb]{1,0,1}}

\def\prf#1{{\bf \begingroup\prfcolor P\ifx#1\empty\else(\endgroup{\I #1}\begingroup\prfcolor)\fi:\endgroup}}
\def\thm#1{{\bf \begingroup\thmcolor T\ifx#1\empty\else(\endgroup{\I #1}\begingroup\thmcolor)\fi:\endgroup}}
\def\lem#1{{\bf \begingroup\lemcolor L\ifx#1\empty\else(\endgroup{\I #1}\begingroup\lemcolor)\fi:\endgroup}}

\def\st{{\rm\ t.ž.\ }}
% ************  TEXT STARTS HERE **************************

\title{Pravděpodobnostní metoda}

\section{Nástroje a odhady}

\thm{Union-bound}

\thm{Čebyšev}

\thm{Markov}

\thm{Černov}

\section{Základní použití}

\thm{Hrubý Ramsey} $\forall k \ge 3: R(k,k) > 2^{k/2 - 1}$.

\prf{} Vezmi $G(n,1/2)$, použij union-bound na jev \uv{$G$ obsahuje kliku nebo
nezávislou velikosti $k$}.

\dfn $m(k)$ bude označovat minimálni počet hran $k$-unif. hypergrafu, který není
$2$-obarvitelný. $m(2) = 3$, $m(3) = 7$.

\thm{Odhad $m(k)$} $m(k) \ge 2^{k-1}$.

\prf{} Vezmi menší náhodný hypergraf. Zvol náhodné obarvení, spočti pravděpodobnost
že existuje jednobarevná (a tedy špatná) hrana.

\section{Náhodné permutace}

\dfn Systém množin $F$ {\I má průnik} pokud mají průnik všechny dvojice množin z $F$.

\thm{Slunečnicové lemma} Nosná množina $n \ge 2k$. Pokud má systém
$k-prvkových$ množin $F$ průnik, tak $|F| \le {n-1 \choose k-1}$.

\lem{} $X = Z_n$, zadefinuj $A_s = \{k,k+1,\dots s+k-1\}$ pro $s \in X$, $n$ buď
větší jak $2k$.  Pak $F$ s průnikem obsahuje nanejvýš $k$ z množin $A_s$.

\prf{} Když tam patří $A_i$, tak tam patří nanejvýš ty, co se s ní pronikají, a
ty se dají rozdělit ještě do exkluzivních dvojic.

\prf{Slunečnicové lemma} Budeme volit náhodné $s$ a náhodnou $\sigma$, co přepermutuje
$A_s$. Pak $P[\sigma(A_s) \in F] \le k/n$.

\section{Linearita střední hodnoty}

\thm{} Existuje turnaj, který má alespoň $n!/2^{n-1}$ Ham. cest.

\thm{} Každý graf s $m$ hranami obsahuje bipartitní graf s $m/2$ hranami.

\section{Alterace}

\thm{Slabý Turán} $\alpha(G) \ge n/2d$.

\prf{}

\thm{Erdös} $\forall k,l \exists G \st \chi(G) > k \& g(G) > l$.

\prf{} Nastav $\epsilon = 1/2l, p = n^{\epsilon-1}$. Střední hodnota počtu cyklů
délky $l$ je $$E[X] \le \sum_{i=3}^l n^{\epsilon i} = o(n).$$
Zvol $n \st E[X] < n/4$. To nám dá $E[X] < 1/2$, použij Markova na $P[X \ge n/2] < 1/2$.

Teď počitej barevnost pomocí největší nezávislé. (Barevné třídy jsou
nezávislé.) $a$ buď $\lceil 3/p \ln n \rceil$, pak máme
$$P[\alpha \ge a] \le {n \choose a} (1-p)^{{a \choose 2}} \le n^a e^{-p{a \choose 2}} =
e^{(\ln n -p(a-1)/2)a}.$$

To jde do nuly, tedy zase máme $P[\alpha] < 1/2$. Smaž jeden vrchol ze všech krátkých cyklů.
Zbyde půlka vrcholů, barevnost odhadni velikostí $\alpha$.

\section{Dependent Random Choice}

\section{Lovászovo lokální lemma}

\thm{Symetrické LLL} Buď $A_1, \dots, A_n \st P[A_i] \le p$ a všech\-ny vý\-stup\-ně v zá\-vi\-slostním
orientovaném grafu jsou nanejvýš $d$. Pokud $ep(d+1) \le 1$, pak 
$$P[\bigcup_{i=1}^n \overline{A_i}] > 0.$$

\prf{lll a} Buď $A_i$, nastav k nim reálné 0-1-proměnné $x_i$. Pokud platí
$$P[A_i] \ge x_i \prod_{(i,j) \in E} (1-x_j), $$
Pak
$$P[\bigcap_i \overline{A_i}] \ge \prod_i (1-x_i) > 0.$$

\prf{} Je třeba počítat $A_i$ (to, co nechceme, aby se stalo) za předpokladu, že
už se nějaka podmnožina $S$ jevů nestala. Indukcí dokazujeme:

$$ P[A_i|\bigcap_{j \in S} \overline{A_j}] \le x_i.$$

%TODO: Doplnit.

\thm{Vážené LLL} $A_i$ jako vždy, $0 \le p \le 1/4$, reálná čísla $t_i$.
S kladnou pravděpodobností se nic zlého nestane, pokud:
\itemize\ibull
\: $P[A_i] \le p^{t_i}$,
\: $\sum_{(i,j) \in E} (2p)^t_j \le t_j /2$.
\endlist

\dfn Obarvení je {\I silně hvězdicové} ($\chi_h$), pokud indukovaný graf každých dvou barev
je hvězda (plus izolované vrcholy).

\thm{} Pro $G$ s $\Delta$ a $m$ hranami, $\chi_h(G) \le 14 \sqrt{\Delta m}$.

\prf{} 

\section{Černov a spol.}

\bye
